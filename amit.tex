\documentclass[11pt]{article}
\pdfoutput=1
\setlength\overfullrule{5pt}
\usepackage[T1]{fontenc}
\usepackage[latin1]{inputenc}
\usepackage[nosort]{cite}
\usepackage{color}
\usepackage[bulletsep]{collref}
\usepackage{graphicx}
%\usepackage{pst-all}
\usepackage{bbm}
\usepackage{amsmath}
\usepackage{amssymb}
\usepackage{subfig}
%\usepackage{slashed}
%\usepackage{psfrag}
%\usepackage{array}
%\usepackage{verbatim}
%\usepackage{feynmp}
\usepackage{multirow}
\usepackage{tikz}
\usepackage{hyperref}



\newcommand{\remark}[1]{{\color{blue} [#1]}}


%%%%%%%%%%%%%%%%%%%%%%%%%%%%%%%%%%%%%%%%%%%%%%%%%%%%%%%%%%%%%%%%%%%%%%%%%%%
\setlength{\topmargin}{-12mm}
\setlength{\evensidemargin}{-2.5mm}
\setlength{\oddsidemargin}{0mm}
\setlength{\textwidth}{165mm}
\setlength{\textheight}{230mm}

\renewcommand{\textfraction}{0.15}
\renewcommand{\floatpagefraction}{0.7}

%\setlength\textwidth{390pt} \setlength\textheight{580pt}
%\addtolength\textwidth{60pt} \addtolength\oddsidemargin{-30pt}
%\addtolength\textheight{70pt} \addtolength\topmargin{-55pt}

%%%%%%%%%%%%%%%%%%%%%%%%%%%%%%%%%%%%%%%%%%%%%%%%%%%%%%%%%%%%%%%%%%%%%%%%%%%
%equation numbers
\makeatletter \@addtoreset{equation}{section} \makeatother
\renewcommand{\theequation}{\thesection.\arabic{equation}}

%%%%%%%%%%%%%%%%%%%%%%%%%%%%%%%%%%%%%%%%%%%%%%%%%%%%%%%%%%%%%%%%%%%%%%%%%%%
%bold maths in section titles
\makeatletter
\let\old@startsection=\@startsection
\let\oldl@section=\l@section
\renewcommand{\@startsection}[6]{\old@startsection{#1}{#2}{#3}{#4}{#5}{#6\mathversion{bold}}}
\renewcommand{\l@section}[2]{\oldl@section{\mathversion{bold}#1}{#2}}
\makeatother

%%%%%%%%%%%%%%%%%%%%%%%%%%%%%%%%%%%%%%%%%%%%%%%%%%%%%%%%%%%%%%%%%%%%%%%%%%%
%small figure captions
\makeatletter
\let\old@makecaption=\@makecaption
\def\@makecaption{\small\old@makecaption}
\makeatother



\renewcommand{\theequation}{\thesection.\arabic{equation}}

\newcommand{\newsection}{
\setcounter{equation}{0}
\section}
\renewcommand{\thefootnote}{\arabic{footnote}}
\setcounter{footnote}{0}




%%%%%%%%%%%%%%%%%%%%%%%%%%%%%%%%%%%%%%%%%%%%%%%%%%%%%%%%%%%%%%%%%%%%%%%%%%%
% slanted greek caps

\let\oldPhi=\Phi
\let\oldPsi=\Psi
\let\oldGamma=\Gamma
\let\oldDelta=\Delta
\let\oldSigma=\Sigma
\let\oldTheta=\Theta
\let\oldPi=\Pi
\let\oldUpsilon=\Upsilon
\renewcommand{\Phi}{\mathnormal{\oldPhi}}
\renewcommand{\Psi}{\mathnormal{\oldPsi}}
\renewcommand{\Gamma}{\mathnormal{\oldGamma}}
\renewcommand{\Sigma}{\mathnormal{\oldSigma}}
\renewcommand{\Delta}{\mathnormal{\oldDelta}}
\renewcommand{\Theta}{\mathnormal{\oldTheta}}
\renewcommand{\Pi}{\mathnormal{\oldPi}}
\renewcommand{\Upsilon}{\mathnormal{\oldUpsilon}}


%%%%%%%%%%%%%%%%%%%%%%%%%%%%%%%%%%%%%%%%%%%%%%%%%%%%%%%%%%%%%%%%%%%%%%%%%%%%%%%%
% symbols

\newcommand{\superN}{\mathcal{N}}
\newcommand{\Action}{\mathcal{S}}
\newcommand{\Lagr}{\mathcal{L}}
\newcommand{\Ham}{\mathcal{H}}
\newcommand{\dil}{\mathcal{D}}
\newcommand{\gym}{g\indups{YM}}
\newcommand{\Op}{\mathcal{O}}
\newcommand{\tr}{\mathop{\mathrm{tr}}}
\newcommand{\Trace}{\mathop{\mathrm{Tr}}}
\newcommand{\str}{\mathop{\mathrm{str}}}
\newcommand{\diag}{\mathop{\mathrm{diag}}}
\newcommand{\sign}{\mathop{\mathrm{sign}}}
\renewcommand{\Re}{\mathop{\mathrm{Re}}}
\renewcommand{\Im}{\mathop{\mathrm{Im}}}
\newcommand{\fld}[1]{\mathcal{#1}}
\newcommand{\imag}{\mathbbm{i}}

\newcommand{\order}{\mathcal{O}}

\newcommand{\hateq}{\mathrel{\widehat=}}
\newcommand{\trans}{{\scriptscriptstyle\mathrm{T}}}

\newcommand{\cdott}{\mathord{\cdot}}
\newcommand{\contour}{\mathcal{C}}
\newcommand{\Naturals}{\mathbbm{N}}
\newcommand{\Integers}{\mathbbm{Z}}
\newcommand{\Reals}{\mathbbm{R}}
\newcommand{\Complex}{\mathbbm{C}}
\newcommand{\Hyp}{\mathbbm{H}}
\newcommand{\Sphere}{\mathrm{S}}  % {\mathbbm{S}}
\newcommand{\AdS}{\mathrm{AdS}}
\newcommand{\shell}{\bigcirc\!\!\!\!\!\!\!~\circ}


%%%%%%%%%%%%%%%%%%%%%%%%%%%%%%%%%%%%%%%%%%%%%%%%%%%%%%%%%%%%%%%%%%%%%%%%%%%%%%%%
% fractions and stuff

\ifx\genfrac\sdflkaj
\newcommand{\atopfrac}[2]{{{#1}\above0pt{#2}}}
\else
\newcommand{\atopfrac}[2]{\genfrac{}{}{0pt}{}{#1}{#2}}
\fi
\newcommand{\sfrac}[2]{{\textstyle\frac{#1}{#2}}}
\newcommand{\half}{\sfrac{1}{2}}
\newcommand{\ihalf}{\sfrac{i}{2}}
\newcommand{\p}{\partial}\newcommand{\quarter}{\sfrac{1}{4}}
\newcommand{\iquarter}{\sfrac{i}{4}}
\newcommand{\Half}{\frac{1}{2}}
\newcommand{\iHalf}{\frac{i}{2}}
\newcommand{\Quarter}{\frac{1}{4}}

\newcommand{\X}{\mathbb{X}}
\newcommand{\J}{\mathbb{J}}
\newcommand{\A}{\mathbb{A}}
\newcommand{\HH}{\mathbb{H}}

%%%%%%%%%%%%%%%%%%%%%%%%%%%%%%%%%%%%%%%%%%%%%%%%%%%%%%%%%%%%%%%%%%%%%%%%%%%%%%%%
% styles

\newcommand{\indup}[1]{_{\mathrm{#1}}}
\newcommand{\indups}[1]{_{\mathrm{\scriptscriptstyle #1}}}
\newcommand{\supup}[1]{^{\mathrm{#1}}}
\newcommand{\rep}[1]{{\mathbf{#1}}}
\newcommand{\matr}[2]{\left(\begin{array}{#1}#2\end{array}\right)}
\newcommand{\alg}[1]{\mathfrak{#1}}
\newcommand{\grp}[1]{\mathrm{#1}}

\newcommand{\grO}{\grp{O}}
\newcommand{\grU}{\grp{U}}
\newcommand{\grSU}{\grp{SU}}
\newcommand{\grSO}{\grp{SO}}
\newcommand{\grSp}{\grp{Sp}}
\newcommand{\grSL}{\grp{SL}}
\newcommand{\grPSU}{\grp{PSU}}
\newcommand{\grOsp}{\grp{Osp}}
\newcommand{\algSU}{\alg{su}}
\newcommand{\algSO}{\alg{so}}
\newcommand{\algPSU}{\alg{psu}}


%%%%%%%%%%%%%%%%%%%%%%%%%%%%%%%%%%%%%%%%%%%%%%%%%%%%%%%%%%%%%%%%%%%%%%%%%%%%%%%%
%brackets
\newcommand{\brk}[1]{(#1)}
\newcommand{\lrbrk}[1]{\left(#1\right)}
\newcommand{\bigbrk}[1]{\bigl(#1\bigr)}
\newcommand{\Bigbrk}[1]{\Bigl(#1\Bigr)}
\newcommand{\biggbrk}[1]{\biggl(#1\biggr)}
\newcommand{\Biggbrk}[1]{\Biggl(#1\Biggr)}
\newcommand{\sbrk}[1]{[#1]}
\newcommand{\lrsbrk}[1]{\left[#1\right]}
\newcommand{\bigsbrk}[1]{\bigl[#1\bigr]}
\newcommand{\Bigsbrk}[1]{\Bigl[#1\Bigr]}
\newcommand{\biggsbrk}[1]{\biggl[#1\biggr]}
\newcommand{\Biggsbrk}[1]{\Biggl[#1\Biggr]}
\newcommand{\ket}[1]{\mathopen{|}#1\mathclose{\rangle}}
\newcommand{\bra}[1]{\mathopen{\langle}#1\mathclose{|}}
\newcommand{\braket}[2]{\mathopen{\langle}#1|#2\mathclose{\rangle}}
\newcommand{\vev}[1]{\langle#1\rangle}
\newcommand{\normord}[1]{\mathopen{:}#1\mathclose{:}}
\newcommand{\lrvev}[1]{\left\langle#1\right\rangle}
\newcommand{\bigvev}[1]{\bigl\langle#1\bigr\rangle}
\newcommand{\Bigvev}[1]{\Bigl\langle#1\Bigr\rangle}
\newcommand{\biggvev}[1]{\biggl\langle#1\biggr\rangle}
\newcommand{\bigcomm}[2]{\big[#1,#2\big]}
\newcommand{\comm}[2]{[#1,#2]}
\newcommand{\lrcomm}[2]{\left[#1,#2\right]}
\newcommand{\acomm}[2]{\{#1,#2\}}
\newcommand{\bigacomm}[2]{\big\{#1,#2\big\}}
\newcommand{\gcomm}[2]{[#1,#2\}}
\newcommand{\lrabs}[1]{\left|#1\right|}
\newcommand{\abs}[1]{{|#1|}}
\newcommand{\bigabs}[1]{\bigl|#1\bigr|}
\newcommand{\Bigabs}[1]{\Bigl|#1\Bigr|}
\newcommand{\bigeval}[1]{#1\big|}
\newcommand{\eval}[1]{#1|}
\newcommand{\lreval}[1]{\left.#1\right|}
\newcommand{\set}[1]{\{#1\}}
\newcommand{\pint}{\makebox[0pt][l]{\hspace{3.4pt}$-$}\int}
\newcommand{\grade}[1]{[#1]}
\newcommand{\units}[1]{[#1]}


%%%%%%%%%%%%%%%%%%%%%%%%%%%%%%%%%%%%%%%%%%%%%%%%%%%%%%%%%%%%%%%%%%%%%%%%%%%%%%%%
% equations

\newcommand{\nn}{\nonumber}
\newcommand{\nln}{\nonumber\\}
\newcommand{\nl}[1][0pt]{\nonumber\\[#1]&\hspace{-4\arraycolsep}&\mathord{}}
\newcommand{\nlnum}{\\&\hspace{-4\arraycolsep}&\mathord{}}
\newcommand{\earel}[1]{\mathrel{}&\hspace{-2\arraycolsep}#1\hspace{-2\arraycolsep}&\mathrel{}}
\newcommand{\eq}{\earel{=}}

%%%%%%%%%%%%%%%%%%%%%%%%%%%%%%%%%%%%%%%%%%%%%%%%%%%%%%%%%%%%%%%%%%%%%%%%%%%%%%%%
% shorter equations

\def\[{\begin{equation}}
\def\]{\end{equation}}

%%%%%%%%%%%%%%%%%%%%%%%%%%%%%%%%%%%%%%%%%%%%%%%%%%%%%%%%%%%%%%%%%%%%%%%%%%%%%%%%
% references
\makeatletter
\def\mr@ignsp#1 {\ifx\:#1\@empty\else #1\expandafter\mr@ignsp\fi}%
\newcommand{\multiref}[1]{\begingroup%\let\protect\string%
\xdef\mr@no@sparg{\expandafter\mr@ignsp#1 \: }%
\def\mr@comma{}%
\@for\mr@refs:=\mr@no@sparg\do{\mr@comma\def\mr@comma{,}\ref{\mr@refs}}%
\endgroup}
\makeatother

\newcommand{\hypref}[2]{\ifx\href\asklfhas #2\else\href{#1}{#2}\fi}
\newcommand{\Secref}[1]{Section~\multiref{#1}}
\newcommand{\secref}[1]{Sec.~\multiref{#1}}
\newcommand{\Appref}[1]{Appendix~\multiref{#1}}
\newcommand{\appref}[1]{App.~\multiref{#1}}
\newcommand{\Tabref}[1]{Table~\multiref{#1}}
\newcommand{\tabref}[1]{Tab.~\multiref{#1}}
\newcommand{\Figref}[1]{Figure~\multiref{#1}}
\newcommand{\figref}[1]{Fig.~\multiref{#1}}
\renewcommand{\eqref}[1]{(\multiref{#1})}


\newcommand{\bibtitle}[1]{\emph{#1}}
% \newcommand{\hepth}[1]{\texttt{hep-th/#1}}
% \newcommand{\condmat}[1]{\texttt{cond-mat/#1}}


%%%%%%%%%%%%%%%%%%%%%%%%%%%%%%%%%%%%%%%%%%%%%%%%%%%%%%%%%%%%%%%%%%%%%%%%%%%%%%%%
%arxiv links

\ifx\href\asklfhas\newcommand{\href}[2]{#2}\fi
\newcommand{\arxivno}[1]{\href{http://arxiv.org/abs/#1}{#1}}

%%%%%%%%%%%%%%%%%%%%%%%%%%%%%%%%%%%%%%%%%%%%%%%%%%%%%%%%%%%%%%%%%%%%%%%%%%%%%%%%
% others
\newcommand{\comma}{\quad,\quad}
\newcommand{\unit}{\mathbbm{1}}

\newcommand{\levi}{\epsilon}
\newcommand{\energy}{\varepsilon}
\newcommand{\eps}{\varepsilon}

\newcommand{\be}{\begin{eqnarray}}
\newcommand{\ee}{\end{eqnarray}}

\DeclareMathOperator{\sech}{sech}
\DeclareMathOperator{\csch}{csch}
\DeclareMathOperator{\arcsinh}{arcsinh}
\DeclareMathOperator{\arccosh}{arccosh}
\DeclareMathOperator{\arctanh}{arctanh}
\DeclareMathOperator{\arcsech}{arcsech}
\DeclareMathOperator{\arccsch}{arccsch}
\DeclareMathOperator{\arccoth}{arccoth}

\newcommand{\Elliptic}[1]{\mathbbm{#1}}
\newcommand{\EllipticE}{\Elliptic{E}}
\newcommand{\EllipticF}{\Elliptic{F}}
\newcommand{\EllipticK}{\Elliptic{K}}
\newcommand{\EllipticPi}{\oldPi}
\DeclareMathOperator{\JacobiAM}{am}

\DeclareMathOperator{\JacobiCN}{cn}
\DeclareMathOperator{\JacobiDN}{dn}
\DeclareMathOperator{\JacobiSN}{sn}

\DeclareMathOperator{\JacobiDC}{dc}
\DeclareMathOperator{\JacobiSC}{sc}
\DeclareMathOperator{\JacobiNC}{nc}

\DeclareMathOperator{\JacobiSD}{sd}
\DeclareMathOperator{\JacobiCD}{cd}
\DeclareMathOperator{\JacobiND}{nd}

\DeclareMathOperator{\JacobiDS}{ds}
\DeclareMathOperator{\JacobiCS}{cs}
\DeclareMathOperator{\JacobiNS}{ns}


\newcommand{\dalpha}{{\dot{\alpha}}}
\newcommand{\dbeta}{{\dot{\beta}}}
\newcommand{\dgamma}{{\dot{\gamma}}}
\newcommand{\ddelta}{{\dot{\delta}}}

\newcommand{\btheta}{{\bar{\theta}}}

\newcommand{\um}{{\underline{m}}}
\newcommand{\un}{{\underline{n}}}
\newcommand{\uk}{{\underline{k}}}
\newcommand{\umu}{{\underline{\mu}}}
\newcommand{\ua}{{\underline{a}}}
\newcommand{\ualpha}{{\underline{\alpha}}}

\newcommand{\cl}{{\mathrm{cl}}}
\newcommand{\nt}{{\mathrm{int}}}

\newcommand{\BIG}{\textcolor[rgb]{0.2472, 0.24, 0.6}{big}}
\newcommand{\SMALL}{\textcolor[rgb]{0.6, 0.24, 0.442893}{small}}



%%%%%%%%%%%%%%%%%%%%%%%%%%%%%%%%%%%%%%%%%%%%%%%%%%%%%%%%%%%%%%%%%%%%%%%%%%%%%%%%
%%%%%%%%%%%%%%%%%%%%%%%%%%%%%%%%%%%%%%%%%%%%%%%%%%%%%%%%%%%%%%%%%%%%%%%%%%%%%%%
\begin{document}


\thispagestyle{empty}
\begin{flushright}\footnotesize
%\texttt{NORDITA-2015-XXX}\\
\end{flushright}
\vspace{1cm}

\begin{center}%
{\Large\textbf{\mathversion{bold}%
Adding probe flavors to $\mathcal{N}=2^*$ using D7 branes
}\par}

\vspace{1.5cm}
%
%\textrm{Amit Dekel} \vspace{8mm} \\
%\textit{%
%Nordita, KTH Royal Institute of Technology and Stockholm University, \\
%Roslagstullsbacken 23, SE-106 91 Stockholm, Sweden
%} \\


%\texttt{\\ amit.dekel@nordita.org}



%%%%%%%%
\par\vspace{14mm}

\textbf{Abstract} \vspace{5mm}

\begin{minipage}{14cm}


\end{minipage}

\end{center}





%%%%%%%%%%%%%%%%%%%%%%%%%%%%%%%%%%%%%%%%%%%%%%%%%%%%%%%%%%%%%%%%%%%%%%%%%%%
%%%%%%%%%%%%%%%%%%%%%%%%%%%%%%%%%%%%%%%%%%%%%%%%%%%%%%%%%%%%%%%%%%%%%%%%%%%
\newpage

\tableofcontents


\bigskip
\noindent\hrulefill
\bigskip


%%%%%%%%%%%%%%%%%%%%%%%%%%%%%%%%%%%%%%%%%%%%%%%%%%%%%%%%%%%%%%%%%%%%%%%%%%%
%%%%%%%%%%%%%%%%%%%%%%%%%%%%%%%%%%%%%%%%%%%%%%%%%%%%%%%%%%%%%%%%%%%%%%%%%%%
\section{Introduction}\label{sec:intro}


\section{Pilch-Warner geometry}\label{sec:PWB}

In this section we shall describe the Pilch-Warner solution to the type IIB supergravity equations, which are given in appendix \ref{app:SUGRAEQ}.
We start be introducing the various background fields which are all non-trivial for this solution.
Our notation and conventions are summarized in appendix \ref{app:notation}.
Afterwards, we introduce the Killing spinors for the background which were found first in \cite{Pilch:2003jg}.

\subsection{The background fields}
In this subsection we quote the expressions for the filed content of the Pilch-Warner solution.
We follow mostly the conventions of \cite{Pilch:2003jg} and \cite{Buchel:2000cn}.
\begin{itemize}
  \item \textbf{The metric}\\
The metric (in the Einstein frame) is given by
\begin{align}\label{eq:PWmetric}
ds_E^2 =
\Omega^2 dx_\mu dx^\mu -\left(
V_r^2 dr^2 + V_\theta^2 d\theta^2 + V_1^2 \sigma_1^2 + V_{23}^2 (\sigma_2^2 + \sigma_3^2) + V_\phi^2 d\phi^2\right),
\end{align}
where we use the mostly minus convention.
The various coefficients are given by
\begin{align}
& \Omega = \frac{c^{1/8} A^{1/4} X_1^{1/8} X_2^{1/8}}{(c^2 - 1)^{1/2}},\quad
V_r = \frac{c^{1/8}X_1^{1/8} X_2^{1/8}}{A^{1/12}},\quad
V_\theta = \frac{X_1^{1/8} X_2^{1/8}}{c^{3/8}A^{1/4}},\nonumber\\
&
V_1 = \frac{A^{1/4}X_1^{1/8} }{c^{3/8}X_2^{3/8}},\quad
V_{23} = \frac{c^{1/8}A^{1/4}X_2^{1/8} }{X_1^{3/8}},\quad
V_\phi = \frac{c^{1/8}X_1^{1/8} }{A^{1/4}X_2^{3/8}},
\end{align}
and
\begin{align}
X_1 = & \cos^2\theta + cA  \sin^2\theta,\nonumber\\
X_2 = & c \cos^2\theta + A  \sin^2\theta.
\end{align}
$c$ and $A$ are functions of the radial coordinate $r$ which satisfy
\begin{align}
A  = c+(c^2 - 1)\frac{1}{2}\ln\left(\frac{c-1}{c+1}\right),
\end{align}
and
\begin{align}
\frac{dc}{dr} = A^{2/3}(1-c^2),\quad
\frac{d A}{dr} = 2 A^{2/3}\left(1 - c A\right).
\end{align}
Finally, $\sigma_i(\alpha,\beta,\psi) = \tr(g^{-1}\tau_i d g)$ are the $SU(2)$ left invariant forms parameterizing $S^3$, where $\tau_i$ are the Pauli matrices. The one forms satisfy the relation $d\sigma_i  = \epsilon_{i j k} \sigma_j \wedge \sigma_k$ (notice that this convention is different than the one used for example in \cite{Buchel:2000cn}, where $d\sigma_i  = -\epsilon_{i j k} \sigma_j \wedge \sigma_k$).
For example, we can parameterize them using the Euler angles of $S^3$
\begin{align}
\sigma_1 = &\frac{1}{2}\left(\sin\alpha d \psi -\cos\alpha \sin\psi d\beta\right),\nonumber\\
\sigma_2 = &-\frac{1}{2}\left(\cos\alpha d \psi +\sin\alpha \sin\psi d\beta\right),\nonumber\\
\sigma_3 = &\frac{1}{2}\left(d \alpha +\cos\psi d\beta\right),
\end{align}
$0 \leq \alpha \leq 2\pi$, $0 \leq \beta \leq 2\pi$, $0 \leq \psi \leq \pi$.
The deformed five-sphere part of the metric has an $SU(2)\times U(1)^2$ symmetry, where the $U(1)$'s correspond to $\phi$ translation and a rotation between $\sigma_1$ and $\sigma_2$.
%
%
%
%The Euclidean Einstein frame metric for the Pilch-Warner background is given by \cite{Pilch:2000ue,Buchel:2000cn}
%\begin{align}\label{eq:PWmetric}
%ds_E^2 =
%\frac{(c X_1 X_2)^{1/4}}{\rho^3}\bigg[
%&
%\frac{k^2 A }{c^2-1}dx_{||}^2 - \frac{L^2}{A (c^2-1)^2}dc^2\nonumber\\
%&-L^2\bigg(
%\frac{1}{c}d\theta^2
%+\frac{\sin^2\theta}{X_2}d \phi^2
%+A \cos^2\theta\left(
%\frac{1}{c X_2}\sigma_3^2
%+\frac{1}{X_1}(\sigma_1^2 + \sigma_2^2)\right)
%\bigg)
%\bigg],
%\end{align}
%where
%\begin{align}
%X_1 = & \cos^2\theta + cA  \sin^2\theta,\nonumber\\
%X_2 = & c \cos^2\theta + A  \sin^2\theta.
%\end{align}
%$\sigma_i(\alpha,\beta,\psi) = \tr(g^{-1}\tau_i d g)$ are the $SU(2)$ left invariant forms parameterizing $S^3$, and $\tau_i$ are the Pauli matrices. The one forms satisfy the relation $d\sigma_i  = \epsilon_{i j k} \sigma_j \wedge \sigma_k$ (notice that this convention is different than the one used for example in \cite{Buchel:2000cn}, where $d\sigma_i  = -\epsilon_{i j k} \sigma_j \wedge \sigma_k$).
%For example, we can parameterize them using the Euler angles of $S^3$
%\begin{align}
%\sigma_1 = &\frac{1}{2}\left(\sin\alpha d \psi -\cos\alpha \sin\psi d\beta\right),\nonumber\\
%\sigma_2 = &-\frac{1}{2}\left(\cos\alpha d \psi +\sin\alpha \sin\psi d\beta\right),\nonumber\\
%\sigma_3 = &\frac{1}{2}\left(d \alpha +\cos\psi d\beta\right),
%\end{align}
%$0 \leq \alpha \leq 2\pi$, $0 \leq \beta \leq 2\pi$, $0 \leq \psi \leq \pi$.
%The deformed five-sphere part of the metric has an $SU(2)\times U(1)^2$ symmetry, where the $U(1)$'s correspond to $\phi$ translation and a rotation between $\sigma_1$ and $\sigma_2$.
%$\rho$ is a function of the radial coordinate $c$, given by
%\begin{align}
%A (c) = c+(c^2 - 1)\left(\gamma + \frac{1}{2}\ln\left(\frac{c-1}{c+1}\right)\right),
%\end{align}
%where $\gamma$ is an integration constant.
%Different signs of $\gamma$ correspond to different asymptotics. We shall be  interested in the $\gamma = 0$  case, which corresponds to pure $\mathcal{N}=2$ flow with vanishing scalar VEV \cite{Pilch:2000ue}
%The parameter k is proportional to the symmetry-breaking mass perturbation, $k = M L$ (?) \cite{Buchel:2000cn}.
%$L$ is a parameter related to the radius in the near boundary limit.


The boundary of the Pilch-Warner geometry is located at $c = 1$.
Defining the coordinate $c = 1 + \frac{z^2}{2}$, and expanding around $z = 0$, we get the $\AdS_5$ metric
\begin{align}
ds_E^2 =
\frac{dx^2 - dz^2}{z^2} + O(z^{0}).
\end{align}
Thus, close to the boundary, $z$ plays the role of the familiar radial AdS coordinate in Poincare coordinates.
Similarly, the second part reduces to the $S^5$ metric,
\begin{align}
ds_{S^5}^2 =
-\left(d\theta^2 + \sin^2\theta d\phi^2 + \cos^2\theta\left(\sigma_1^2 + \sigma_2^2 + \sigma_3^2\right)\right) + O(z^{1}).
\end{align}

%The other nontrivial fields in the PW solutions are :
%\begin{itemize}
%  \item \textbf{Dilaton-Axion field}\\
%The field is given by
%\begin{align}
%V = f\left(
%  \begin{array}{cc}
%    1 & B \\
%    B^* & 1 \\
%  \end{array}
%\right),
%\end{align}
%where
%\begin{align}
%f = \cosh\left(\frac{1}{4}\ln\left(\frac{c X_1}{X_2}\right)\right),\quad
%f B = e^{2 i \phi}\sinh\left(\frac{1}{4}\ln\left(\frac{c X_1}{X_2}\right)\right).
%\end{align}
%Notice that $f = \frac{1}{\sqrt{1- |B|^2}}$, or equivalently $B = e^{2 i \phi}\sqrt{1-\frac{1}{f^2}}$.
%Using $\tau = C_0 + i e^{-\Phi} = i\left(\frac{1-B}{1+B}\right)$ \cite{Evans:2000ct}
%we find $C_0 = \frac{\left(-1+c^2\right) \sin(2 \phi )}{1+c^2+\left(-1+c^2\right) \cos(2 \phi )}$ and
%$e^{-\Phi} = \frac{2 c}{1+c^2+\left(-1+c^2\right) \cos(2 \phi )}$
%where we also used $f = \frac{c+1}{2\sqrt{c}}$ which we get when setting $\theta = \pi/2$.
%By further setting $\phi = \pi/2$ we get $C_0 = 0$ and $e^{-\Phi} = c$.
%(Notice the difference compared to \cite{Buchel:2000cn} where, $\tau = C_0 + i e^{-\Phi} = \frac{\theta}{2\pi} + \frac{i}{g_s}\left(\frac{1+B}{1-B}\right)$, so to get $e^{-\Phi} = c$ one should have $\phi = 0$).
%
%Comparing to standard textbook definition (e.g. Polchinski (12.1.30b) or BBS (8.66))
%the matrix should be similar to the matrix
%\begin{align}
%e^{\Phi }\left(\begin{array}{cc}
% |\tau|^2 & -C_0\\
% -C_0 & 1
%\end{array}
%\right).
%\end{align}
%From this relation we see that
%\begin{align}
%f = \cosh\Phi + C_0^2 \frac{e^{\Phi}}{2}.
%\end{align}
%If $C_0 = 0$ and $f = \cosh(\frac{1}{2}\ln c)$, we have $e^{\pm 2\Phi} = c$.
%
%For the D5-brane case $\phi$ should not be a constant and the axion field should not vanish in principle. Should we also consider the $\theta_s$ parameter?
%%
%%
%%\begin{align}
%%e^\phi = f,\quad
%%C_0 = \frac{\sqrt{f^2 - 1}}{f} = |B|,
%%\end{align}
%%then we have $\tau = |B| + \frac{i}{f}$.
%%On the other hand $\tau = C_0 + i e^{-\phi} = i\left(\frac{1-B}{1+B}\right)$ in \cite{Evans:2000ct} and $\tau = C_0 + i e^{-\phi} = \frac{\theta}{2\pi} + \frac{i}{g_s}\left(\frac{1+B}{1-B}\right)$ in \cite{Buchel:2000cn}.
%  \item \textbf{Three form}\\
%  The antisymmetric three form is given by $F_{(3)} = d A_{(2)}$ where
%\begin{align}
%A_{(2)} = & e^{i \phi}\left(a_1 d\theta \wedge \sigma_1 + a_2 \sigma_2 \wedge \sigma_3 + a_3 \sigma_1 \wedge d\phi\right),\nonumber\\
%a_1 = &- i \frac{4}{g^2}\tanh 2\chi \cos \theta = - i L^2 \frac{\sqrt{c^2-1}}{c}\cos\theta,\nonumber\\
%a_2 =  &i \frac{4}{g^2}A  \frac{\sinh 2 \chi}{X_1}\sin \theta \cos^2 \theta = i A  \frac{\sqrt{c^2-1}}{X_1}\sin \theta \cos^2 \theta,\nonumber\\
%a_3 =  & -\frac{4}{g^2}\frac{\sinh 2 \chi}{X_2}\sin \theta \cos^2 \theta = -\frac{\sqrt{c^2-1}}{X_2}\sin \theta \cos^2 \theta.
%\end{align}
%Here we used $g = 2/L$  and $c = \cosh 2\chi$.
%For the $a_3$ equation we use the sign as given in \cite{Pilch:2003jg} which is opposite to the one in \cite{Pilch:2000ue}.
%$A_{(2)}$ indices are only on the sphere part, and when $\theta = \pi/2$ in vanishes.
%
%Following \cite{Buchel:2000cn}, one can separate the NSNS and RR contributions in $A_{(2)} = C_{(2)} + i B_{(2)}$.
%%Is $F_{(3)} = d A_{(2)} = d C_2 - C_0 d B_2$ (see (8.56) in BBS)?
%
%
%  \item \textbf{Five form}\\
%  The five form is given by $F_{(5)} = \mathcal{F} + \ast \mathcal{F}$, where
%\begin{align}
%\mathcal{F} = & dx^0 \wedge dx^1 \wedge dx^2 \wedge dx^3 \wedge dw,\nonumber\\
%w = & \frac{k^4 A  X_1}{g_s \sinh^4 2\chi} = \frac{k^4 A  X_1}{g_s (c^2 - 1)^2}.
%\end{align}
%$(L = (\alpha')^{1/2}(4\pi g_s N)^{1/4})$.
%Here we used the conventions of \cite{Buchel:2000cn} and \cite{Evans:2000ct} where they claim there is a typo in \cite{Pilch:2000ue} in the expression for $w$.
%\end{itemize}
%
%Notice that \cite{Buchel:2000cn}
%\begin{align}
%F_{(5)} = d A_{(4)} - \frac{1}{8} \Im (A_{(2)}\wedge F_{(3)}^*),\quad
%A_{(4)} = \frac{1}{4}\left(C_{(4)} + \frac{1}{2}B_{(2)}\wedge C_{(2)}\right).
%\end{align}

  \item \textbf{Dilaton-Axion field}\\
The field is given by
\begin{align}
V = f\left(
  \begin{array}{cc}
    1 & B \\
    B^* & 1 \\
  \end{array}
\right),
\end{align}
where
\begin{align}
f = \cosh\left(\frac{1}{4}\ln\left(\frac{c X_1}{X_2}\right)\right),\quad
f B = e^{2 i \phi}\sinh\left(\frac{1}{4}\ln\left(\frac{c X_1}{X_2}\right)\right).
\end{align}

Using the relation $C_{(0)} + i e^{-\Phi} = i\frac{B+1}{B-1}$
we find
\begin{align}
C_0 = \frac{2 b \sin 2\phi}{1-2 b \cos 2\phi + b^2},\quad
e^{-\Phi} = \frac{b^2 - 1}{1-2 b \cos 2\phi + b^2},
\end{align}
where $b = \tanh\left(\frac{1}{4}\ln\left(\frac{c X_1}{X_2}\right)\right)$.


  \item \textbf{Three form}\\
  The antisymmetric three form is given by $F_{(3)} = d A_{(2)} = d\left(C_{(2)} + i B_{(2)}\right)$ where \cite{Pilch:2000ue,Pilch:2003jg}
\begin{align}
A_{(2)} = & e^{i \phi}\left(a_1 d\theta \wedge \sigma_1 + a_2 \sigma_2 \wedge \sigma_3 + a_3 \sigma_1 \wedge d\phi\right),\nonumber\\
a_1 = & - i \frac{\sqrt{c^2-1}}{c}\cos\theta,\nonumber\\
a_2 =  & i A  \frac{\sqrt{c^2-1}}{X_1}\sin \theta \cos^2 \theta,\nonumber\\
a_3 =  &  -\frac{\sqrt{c^2-1}}{X_2}\sin \theta \cos^2 \theta.
\end{align}
%The first equality is cited from \cite{Pilch:2000ue} where the second one from \cite{Pilch:2003jg}.
%The relation between the two is $c = \cosh 2\chi$ (and we set $g = 2$ ? which is weird), I think the second equality is correct since it fits with the supergravity equations I checked.
%For the $a_3$ equation we use the sign as given in \cite{Pilch:2003jg} which is opposite to the one in \cite{Pilch:2000ue}.
$A_{(2)}$ has indices only on the sphere part, and when $\theta = \pi/2$ it vanishes, however notice that $G_{\mu\nu\rho}G^{* \mu\nu\rho}$ does not.

The explicit RR and NSNS potentials are given by
\begin{align}
C_{(2)} = & \cos\phi (a_3 \sigma_1 \wedge d\phi) + i \sin\phi \left(a_1 d\theta \wedge \sigma_1 + a_2 \sigma_2 \wedge \sigma_3\right),\nonumber\\
B_{(2)} = & \sin\phi (a_3 \sigma_1 \wedge d\phi) - i\cos\phi \left(a_1 d\theta \wedge \sigma_1 + a_2 \sigma_2 \wedge \sigma_3\right),
\end{align}
respectively.


  \item \textbf{Five form}\\
  The five form is given by $F_{(5)} = \mathcal{F} + \ast \mathcal{F}$, where\footnote{Notice that we use $c^{1/2}$ in the denominator in contrast to $c$ which appears in \cite{Pilch:2003jg} without the square root.}
\begin{align}
\mathcal{F} = & dx^0 \wedge dx^1 \wedge dx^2 \wedge dx^3 \wedge dw,\nonumber\\
w = & \frac{\Omega^4 X_1^{1/2}}{4 c ^{1/2} X_2^{1/2}    }.
\end{align}
%$(L = (\alpha')^{1/2}(4\pi g_s N)^{1/4})$.
%Here we used the conventions of \cite{Buchel:2000cn} and \cite{Evans:2000ct} where they claim there is a typo in \cite{Pilch:2000ue} in the expression for $w$.
By our conventions $F_{(5)} = \frac{1}{4}d C_{(4)}$.

  \item \textbf{Higher RR fields}\\
$C_{(6)}$ and $C_{(8)}$ are fixed by the following relations
\begin{align}
d C_{(6)} = & -\ast d C_{(2)} - C_{(0)} \ast d B_{(2)} - C_{(4)} \wedge  d B_{(2)},\nonumber\\
d C_{(8)} = & +\ast d C_{(0)} - C_{(6)} \wedge  d B_{(2)}.
\end{align}


\end{itemize}


\subsection{Killing spinors}
The Killing spinors are defined using the vanishing of the supersymmetry transformation of the gravitino and dilatino, given by
%\footnote{With the conventions we use for $F_{(5)}$ it seems the factor of $\frac{1}{480}$ should be replaced with $\frac{1}{1920}$ as in \cite{Skenderis:2002vf}.} \cite{Schwarz:1983qr}
\begin{align}\label{eq:KSE}
& \delta \psi_\mu = D_\mu \epsilon + \frac{i}{480} F_{\rho_1 \rho_2 \rho_3 \rho_4 \rho_5}\gamma^{\rho_1 \rho_2 \rho_3 \rho_4 \rho_5} \gamma_\mu \epsilon + \frac{1}{96}\left(\gamma_\mu{}^{\nu \rho \lambda}G_{\nu \rho \lambda} - 9 \gamma^{\rho \lambda} G_{\mu \rho \lambda}\right)\epsilon^*,\nonumber\\
& \delta \lambda = i P_\mu \gamma^\mu \epsilon^* - \frac{i}{24}G_{\mu \nu \rho}\gamma^{\mu \nu \rho}\epsilon,
\end{align}
where $\epsilon$ is a chiral spinor $\Gamma^{11}  \epsilon = - \epsilon$, ($\Gamma^{11}\equiv \Gamma^1 \Gamma^2..\Gamma^{10}$), and
\begin{align}
D_\mu \epsilon  = \p_\mu \epsilon +\frac{1}{4}\omega_{\mu}{}^{ab}\Gamma_{ab}\epsilon -\frac{i}{2}Q_\mu \epsilon,
\end{align}
where the spin connection is defined as follows
\begin{align}
&\omega_{\mu \nu \rho} = -( \Omega_{\nu \rho \mu } + \Omega_{\nu \mu \rho} + \Omega_{\mu \rho \nu } )\nonumber\\
&\Omega_{\mu \nu \rho} = e^a{}_\rho \p_{[\mu}e_{\nu] a},\nonumber\\
& g_{\mu \nu} = e^a_\mu e^b_\nu \eta_{ab},\quad
E^{\mu a} = g^{\mu \nu }e_\nu^a.
\end{align}
Throughout this section we use the metric in the Einstein frame, unless stated otherwise.
Also, if both indices are upper or lower, the first one will be the curved space and the second one the flat space index.
In order to avoid confusion, we denote the inverse whenever the vielbein has an upper curved space index we denote it by $E$ instead of $e$.
Gamma matrices with Latin indices are constant matrices and with Greek indices are generally not, given by $\gamma_\mu = e_\mu^a \Gamma_a$, we also differ these gamma matrices by using uppercase and lowercase letters to avoid confusion when the index takes a specific value.
We shall use the mostly minus convention where $\eta = \text{diag}(+,-,-,..,-)$.
The fields $P_\mu$, $Q_\mu$ and $G_{\mu \nu \rho}$ are defined as
\begin{align}
P_\mu & = f^2 \p_\mu B,\nonumber\\
Q_\mu & = f^2 \Im(B \p_\mu B^*),\nonumber\\
G_{\mu \nu \rho} & = f\left(F_{\mu \nu \rho} - B F_{\mu \nu \rho}^*\right).
\end{align}

\subsection{Killing spinors in PW background}
The Killing spinors for the PW background were found in \cite{Pilch:2003jg}. In this section we shall repeat the analysis, using the assumptions given in \cite{Pilch:2003jg}, which basically give the ansatz
\begin{align}
\epsilon = e^{i\phi/2}\mathcal{M}(r,\theta)\epsilon_0,
\end{align}
where $\mathcal{M}(r,\theta)$ is a matrix and $\epsilon_0$ is a spinor which depends only on the $su(2)$ cooedinates.
As in \cite{Pilch:2003jg}, we note that for the following combination of Killing spinor equations, the anti-symmetric form do not appear,
\begin{align}
2(\gamma^1\delta\psi_1 + \gamma^{10}\delta\psi_{10}) + ie^{2 i \phi}(\delta\lambda)^*
=
\frac{2 i}{\sin\theta}\left(\frac{A^{2} c^3 X_1^3}{X_2}\right)^{1/8}\Gamma^{10}
\mathcal{P}_-(r,\theta)\epsilon,
\end{align}
where
\begin{align}
\mathcal{P}_\pm(r,\theta) = \frac{1}{2}\left(1 \pm \frac{i}{\sqrt{X_1}}\left(\sqrt{c}A^{1/2}\sin\theta\Gamma^{5,10} + \cos\theta\Gamma^{6,10}\right)\right).
\end{align}
Thus, $\epsilon = \mathcal{P}_+(r,\theta)\epsilon \equiv \epsilon_+$.
Defining
\begin{align}
\cos\lambda = \frac{\cos\theta}{\sqrt{X_1}}, \quad
\sin\lambda = \frac{\sqrt{c}A^{1/2}\sin\theta}{\sqrt{X_1}},
\end{align}
then the projector becomes
\begin{align}
\mathcal{P}_\pm(r,\theta) = \frac{1}{2}\left(1 \pm i \left(\sin\lambda\Gamma^{5,10} + \cos\lambda\Gamma^{6,10}\right)\right) =
O(\lambda) \frac{1}{2}\left(1\pm i\Gamma^{6,10}\right)O^{-1}(\lambda)
\equiv
O(\lambda) \mathcal{P}_{\pm}^{6,10}O^{-1}(\lambda),
\end{align}
where
\begin{align}\label{eq:rotmat56}
O(\lambda)  = \left(\cos\frac{\lambda}{2} - \sin\frac{\lambda}{2}\Gamma^{5 6}\right).
\end{align}

Next we check study the dilatono equation.
We shall use the two conditions
\begin{align}
\Gamma^{11}  \epsilon = - \epsilon,\quad
\Gamma^{10}\epsilon = \left(\frac{i \sqrt{c}A^{1/2}\sin\theta}{\sqrt{X_1}}\Gamma^{5} + \frac{i \cos\theta}{\sqrt{X_1}}\Gamma^{6}\right)\epsilon
\equiv  (c_5 \Gamma^5 + c_6 \Gamma^6)\epsilon.
\end{align}
Using these conditions we get
\begin{align}
\delta \lambda = -i(p_6 - c_6 p_{10})\Gamma^{589}\left(1 + \frac{p_5 - c_5 p_{10}}{p_6 - c_6 p_{10}} \Gamma^{56}\right) \Pi_+ \epsilon,
\end{align}
where
\begin{align}
\Pi_\pm = \frac{1}{2}\left(1\pm i\Gamma^{1234}\left(\sqrt{\frac{X_1}{c X_2}} + \sqrt{\frac{c^2 - 1}{c X_2}}\cos\theta\Gamma^{7,10} *\right)\right),
\end{align}
where we defined $p_a = P_\mu E_a^\mu$.
We further define
\begin{align}
\cos\xi(r,\theta) = \sqrt{\frac{X_1}{c X_2}},\quad
\sin\xi(r,\theta) = -\sqrt{\frac{c^2 - 1}{c X_2}}\cos\theta,
\end{align}
Thus, so far we have seen that the Killing spinor satisfies
\begin{align}
\Pi_+(\xi(r,\theta))\epsilon = 0,\quad
\mathcal{P}_-(\lambda(r,\theta))\epsilon = 0.
\end{align}

Next, we have to check nine more gravitino equations.
We start with the flat directions where the equations are purely algebraic since we assume $\epsilon$ to be independent of $x^\mu$.
Using the two projectors and chirality condition it is straight forward to show that $\delta \psi_1 = ... = \delta \psi_4 = 0$.
Next we consider the $su(2)$ directions. Again, using the projectors and chirality condition we get
\begin{align}
& \delta \psi_7 = \left(\p_{\sigma_1} - i\Gamma^{1234}\left(\cos\lambda\Gamma^{57} - \sin\lambda\Gamma^{67}\right)\right)\epsilon
= \left(\p_{\sigma_1} - i\Gamma^{1234}O(\lambda)\Gamma^{57}O^{-1}(\lambda)\right)\epsilon,\nonumber\\
& \delta \psi_8
= \left(\p_{\sigma_2} + \left(\cos\lambda\Gamma^{58} - \sin\lambda\Gamma^{68}\right)\right)\epsilon
= \left(\p_{\sigma_2} + O(\lambda)\Gamma^{58}O^{-1}(\lambda)\right)\epsilon,\nonumber\\
& \delta \psi_9 = \left(\p_{\sigma_3} + \left(\cos\lambda\Gamma^{59} - \sin\lambda\Gamma^{69}\right)\right)\epsilon
= \left(\p_{\sigma_3} + O(\lambda)\Gamma^{59}O^{-1}(\lambda)\right)\epsilon,\nonumber\\
\end{align}
where by $\p_{\sigma_i}$ we mean the dual of $\sigma_i$. These equation can be summarized in the following form
\begin{align}
\left(\p_{\sigma_i} + O(\lambda)t_i O^{-1}(\lambda)\right)\epsilon = 0,
\end{align}
with
\begin{align}
t_1 = - i\Gamma^{1234}\Gamma^{57},\quad
t_2 = \Gamma^{58},\quad
t_3 = \Gamma^{59}.
\end{align}
It is easy to check that $[\Pi_- \mathcal{P}_+, O(\lambda)t_i O^{-1}(\lambda)] = 0$, so it is convenient to define
\begin{align}
\epsilon \sim \Pi_- \mathcal{P}_+  O(\lambda) \epsilon_0
= \Pi_- O(\lambda) \mathcal{P}_+^{6,10}  \epsilon_0,
\end{align}
so that $\epsilon_0$ depends only on the $su(2)$ coordinates through
\begin{align}\label{eq:su2depnedenceofKS}
\left(\p_{\sigma_i} + t_i \right)\epsilon_0 = 0.
\end{align}
Now we are left with two more equations, $\delta\psi_5 = 0$ and $\delta\psi_6 = 0$.
Defining $\delta\psi_M = (\p_M + \Delta_M) \epsilon$ we have
\begin{align}\label{eq:deltam}
\Delta_M \epsilon %&=
&=\p_M \left(
\frac{1}{16}\ln\left(\frac{4 X_2^3 c^3}{X_1 A^{2} \cos^8\theta}\right)
+\Gamma^{56} \frac{\lambda}{2}
-i\Gamma^{1234} \frac{1}{2}\ln\tan\frac{\xi}{2} \right)\epsilon.
\end{align}
Next we take $\epsilon$ to have the form
\begin{align}
\epsilon
= e^{i\phi/2}O(\lambda) \Pi_-(\xi) M(r,\theta)  \mathcal{P}_+^{6,10}  \epsilon_0,
\end{align}
with $M = \frac{\alpha(r,\theta)}{\sin\frac{\xi}{2}}\Pi^0_+ + \frac{\beta(r,\theta)}{\cos\frac{\xi}{2}}\Pi^0_-$,
$\Pi_-(\xi) = O^*(\xi)\Pi_-^0 O^*(\xi)^{-1}$, $O^*(\xi) = \cos\frac{\xi}{2} + \sin\frac{\xi}{2}\Gamma^{7,10}*$ and $\Pi_\pm^0 = \frac{1}{2}\left(1 \pm i\Gamma^{1234}\right)$.
Because $\frac{1}{2}\p\lambda = \Gamma^{56}  O(\lambda)^{-1} \p O(\lambda)$, the second term in (\ref{eq:deltam}) cancel the $O(\lambda)$ derivative of $\epsilon$, so the Killing spinor equations live us with
\begin{align}\label{eq:reducedKSE}
\p_M(\Pi_-(\xi) M(r,\theta)  ) + \p_M \left(
\frac{1}{16}\ln\left(\frac{4 X_2^3 c^3}{X_1 A^{2} \cos^8\theta}\right)
-i\Gamma^{1234} \frac{1}{2}\ln\tan\frac{\xi}{2} \right)\Pi_-(\xi) M(r,\theta) =0.
\end{align}
We can write $\Pi_-(\xi) M(r,\theta)$ as follows
\begin{align}
\Pi_-(\xi) M(r,\theta)
& = O^*(\xi)\Pi_-^0 O^*(\xi)^{-1}\left(\frac{\alpha}{\sin\frac{\xi}{2}}\Pi^0_+ + \frac{\beta}{\cos\frac{\xi}{2}}\Pi^0_-\right)\nonumber\\
& = O^*(\xi)\left(\cos\frac{\xi}{2}\Pi_-^0  - \sin\frac{\xi}{2}\Gamma^{7,10}*\Pi_+^0 \right)\left(\frac{\alpha}{\sin\frac{\xi}{2}}\Pi^0_+ + \frac{\beta}{\cos\frac{\xi}{2}}\Pi^0_-\right)\nonumber\\
& = \left(\cos\frac{\xi}{2} + \sin\frac{\xi}{2}\Gamma^{7,10}*\right)
\left(\beta\Pi_-^0  - \alpha\Gamma^{7,10}*\Pi_+^0 \right)\nonumber\\
& =
\left(\Pi_-^0\beta\cos\frac{\xi}{2} + \Pi_+^0  \beta\sin\frac{\xi}{2}\Gamma^{7,10}*
- \Pi_-^0\alpha\cos\frac{\xi}{2}\Gamma^{7,10}* + \Pi_+^0\alpha \sin\frac{\xi}{2} \right).
\end{align}
Notice that $i\Gamma^{1234}\Pi_\pm^0 = \pm \Pi_\pm^0$.
Each term should satisfy (\ref{eq:reducedKSE}) independently. It is also easy to see that $\alpha$ and $\beta$ satisfy the same first order linear differential equation
\begin{align}
\p_M\left(\beta\cos\frac{\xi}{2}\right) + \p_M \left(
\frac{1}{16}\ln\left(\frac{4 X_2^3 c^3}{X_1 A^{2} \cos^8\theta}\right)
+\frac{1}{2}\ln\tan\frac{\xi}{2} \right)\beta\cos\frac{\xi}{2} = 0
\end{align}
so we have
\begin{align}
\p_M\ln \left(\beta\left(\frac{X_2^3 c^3}{X_1 A^{2} \cos^8\theta}\right)^{1/16}\sin^{1/2}\xi \right) = 0,
\end{align}
which is solved by
\begin{align}
\beta = \beta_0\left(\frac{X_2^3 c^3 \sin^{8}\xi}{X_1 A^{2} \cos^8\theta}\right)^{-1/16} .
\end{align}
Thus, we find that
\begin{align}
\Pi_-(\xi) M(r,\theta)
& = \left(\frac{X_2^3 c^3 \sin^{8}\xi}{X_1 A^{2} \cos^8\theta}\right)^{-1/16} O^*(\xi)
\Pi_-^0\left(\beta_0  - \alpha_0\Gamma^{7,10}* \right).
\end{align}
In the AdS limit ($c\to 1 + \frac{z^2}{2}$, $z\to 0$) we get
\begin{align}
\Pi_-(\xi) M(r,\theta) \simeq \frac{1}{\sqrt{z}}\Pi_-^0\left(\beta_0  - \alpha_0\Gamma^{7,10}* \right),
\end{align}
so we choose $\alpha_0 = 0$, and the Killing spinor becomes
\begin{align}
\epsilon = e^{i\phi/2}\Omega^{1/2} O(\lambda)O^*(\xi)
\Pi_-^0 \mathcal{P}_+^{6,10}\epsilon_0,
\end{align}
where $\Omega(r,\theta) = \left(\frac{X_1 A^{2} \cos^8\theta}{X_2^3 c^3 \sin^{8}\xi}\right)^{1/8} = \frac{c^{1/8}X_1^{1/8}X_2^{1/8} A^{1/4} }{(c^2-1)^{1/2}}$, is the vielbein of the $M=1,..,4$ coordinates.
Another way to rewrite the solution is the following,
\begin{align}
\epsilon = \frac{e^{i\phi/2}\Omega^{1/2}}{\cos\frac{\xi}{2}}O(\lambda)
%O^*(\xi)\Pi_-^0 O^*(\xi)^{-1}
\Pi_-(\xi)
\mathcal{P}_+^{6,10}\Pi^0_-\epsilon_0,
\end{align}
which coincides with the result of \cite{Pilch:2003jg}.
In the $\theta\to\pi/2, \phi \to 0$ limit we find
\begin{align}
\epsilon(\theta=\pi/2,\phi=0) = \frac{\sqrt{2}c^{1/8}A^{1/4}}{(c^2-1)^{1/4}}
\frac{1}{2}(1-\Gamma^{56})\mathcal{P}_+^{6,10}\Pi^0_-\epsilon_0.
\end{align}
Notice that $\frac{1}{2}(1-\Gamma^{56})$ is not a projector. In this limit $\xi = 0$ and $\lambda = \pi/2$ and $\Pi_-(\xi=0) = \Pi_-^0$.
Also, the AdS limit gives
\begin{align}
\epsilon(c = 1 + \frac{z^2}{2}) = z^{-1/2}O(\theta)\Pi_-^0 \mathcal{P}_+^{6,10}\epsilon_0 + \mathcal{O}(z^{1/2}).
\end{align}
Also, notice that the $\Pi_-^0$ projector in front of $\epsilon_0$ implies that we can take $t_1 = \Gamma^{57}$ instead of $t_1 = -i\Gamma^{1234}\Gamma^{57}$. Alternatively, using the chirality condition and the $\mathcal{P}_+^{6,10}$ projector we can arrive at $t_1 \sim \Gamma^{89}$.



\section{Supersymmetry of the D3 brane solution}

In order to check if our solution is supersymmetric,
we needs to check how the kappa-symmetry projection acts on the Killing spinors which where introduced in section \ref{sec:PWB}.
In this section we shall define the kappa symmetry projector, evaluate it on the D3-brane solution and show that it breaks half of the supersymmetries of the background.


\subsection{Kappa symmetry projection}

The kappa symmetry projection for a given brane is given by \cite{Skenderis:2002vf}
\begin{align}
d^{p+1} \xi \Gamma = - e^{-\Phi} L_{\text{DBI}}^{-1} e^{\mathcal{F}}\wedge X|_{\text{Vol}},
\end{align}
where
\begin{align}
X = \bigoplus_n \gamma_{(2n)} K^n I,
\end{align}
and $|_{\text{Vol}}$ indicates projection to the appropriate volume form as in the LHS.
The operators $K$ and $I$ act on a spinor $\psi$ as $K \psi = \psi^*$ and $I \psi = -i \psi$.
We also defined
\begin{align}
\gamma_{(n)} = \frac{1}{n !}d\xi^{a_n}\wedge ... \wedge d\xi^{a_1} \gamma_{a_1...a_n},
\end{align}
and
\begin{align}
\gamma_{a_1...a_n} = \p_{a_1} X^{\mu_1}...\p_{a_n} X^{\mu_n}\gamma_{\mu_1...\mu_n}.
\end{align}
$\Gamma^2 = 1$ is traceless.
The background supersymmetry which is preserved by the D-brane configuration corresponds to the Killing spinors which are consistent with
\begin{align}
\Gamma \epsilon = -\epsilon.
\end{align}
(other conventions lead to $\Gamma \epsilon = \epsilon$ as in \cite{Skenderis:2002vf} we the mostly plus metric convention is used).



\section{Real analysis}
The Killing spinor is given by
\begin{align}
\epsilon = M \eta = \Omega^{1/2} e^{i\phi/2} O_1(\lambda) O_2(\xi) \eta
= \Omega^{1/2} O_1(\lambda) O_3(\phi) O_2(\xi) \eta
\end{align}
where $O_3(\phi)=\cos\frac{\phi}{2}-\Gamma_{6,10}\sin\frac{\phi}{2}$, where we eliminated the $i$ factor using $i\eta = -\Gamma_{6,10}\eta$.
Now,
\begin{align}
\gamma_{(2n)} K^n I \epsilon
&= \gamma_{(2n)} K^n I \Omega^{1/2} O_1(\lambda) O_3(\phi) O_2(\xi) \eta
= \gamma_{(2n)} K^n \Omega^{1/2} O_1(\lambda) O_3(\phi) O_2(-\xi) i \eta\nonumber\\
&= -\gamma_{(2n)}\Omega^{1/2} O_1(\lambda) O_3(\phi) O_2(-\xi) \Gamma_{6,10} K^n \eta\nonumber\\
&= -M M^{-1}\gamma_{(2n)}\Omega^{1/2} O_1(\lambda) O_3(\phi) O_2(-\xi) \Gamma_{6,10} K^n \eta.
\end{align}
Assuming that $\eta$ is constant with respect to $c$ and $\theta$, we should be interested in
\begin{align}
\pm 1 & = -M^{-1}\gamma_{(2n)}\Omega^{1/2} O_1(\lambda) O_3(\phi) O_2(-\xi) \Gamma_{6,10} K^n \eta\nonumber\\
 &=
  -O_2(-\xi)O_3(-\phi)O_1(-\lambda)\gamma_{(2n)}O_1(\lambda) O_3(\phi) O_2(-\xi) \Gamma_{6,10} K^n \eta.
\end{align}
We shall also use the chirality condition to get rid of $\Gamma_{89}$, $\Gamma_{89}\eta = -\Gamma_{57}\eta$.
\begin{align}
&-O_2(-\xi)O_3(-\phi)O_1(-\lambda)\Gamma_{89}O_1(\lambda) O_3(\phi) O_2(-\xi) \Gamma_{6,10} K^n \eta\nonumber\\
&=
O_2(-\xi)O_3(-\phi)O_1(-\lambda)\Gamma_{57}O_1(-\lambda) O_3(\phi) O_2(\xi) \Gamma_{6,10} K^n \eta.
\end{align}


\section{D7 branes in $\AdS_5\times \Sphere^5$}
Let's assume a probe D7 brane with the worldvolume coordinates $x^\mu,z,\alpha,\beta,\psi$, with $\theta = \theta(z)$ and $\phi$-fixed (to zero).
We also assume no gauge field, thus the nontrivial DBI Lagrangian is given by
\begin{align}
L_{DBI} = \sqrt{|g|} = \frac{\cos ^3 \theta  \sin \psi  \sqrt{z^2 \theta'^2+1}}{8 z^5}.
\end{align}
The EOM give
\begin{align}
\frac{\sin \psi  \cos ^2\theta  \left(z \cos \theta  \left(4 z^2 \theta '^3-z \theta ''+3 \theta '\right)-3 \left(z^2 \theta '^2+1\right) \sin \theta \right)}{8 z^5 \left(z^2 \theta '^2+1\right)^{3/2}} = 0.
\end{align}
Two obvious solutions are $\theta = 0,\pi/2$.
Let's look at the SUSY constraint for supersymmetric solutions.
We should only worry about $\gamma_{(8)}$ since the gauge field is set to zero.
We have
\begin{align}
&\Gamma = L_{DBI}^{-1}\frac{1}{z^5}\frac{\sin\psi\cos^3\theta}{8}\Gamma_{1234}\left(1 + z\theta' \Gamma_{56}\right)\Gamma_{5789}I\nonumber\\
&= -L_{DBI}^{-1}\frac{1}{z^5}\frac{\sin\psi\cos^3\theta}{8}(\Pi_+-\Pi_-)\left(1 + z\theta' \Gamma_{56}\right)(P_+ - P_-).
\end{align}
Let's see how this operator acts on $\epsilon_+$.
\begin{align}
\Gamma \epsilon_+&= -L_{DBI}^{-1}\frac{1}{z^5}\frac{\sin\psi\cos^3\theta}{8}\left(1 + z\theta' \Gamma_{56}\right)(P_+ - P_-)\epsilon_+\nonumber\\
& = -L_{DBI}^{-1}\frac{1}{z^5}\frac{\sin\psi\cos^3\theta}{8}\left(1 + z\theta' \Gamma_{56}\right)(P_+ - P_-)
\sqrt{z}\Gamma^5e^{- \frac{\theta}{2}\Gamma_{56}}e^{- \frac{\theta}{2}\Gamma_{6,10}}\lambda_3\nonumber\\
& = -L_{DBI}^{-1}\frac{1}{z^5}\frac{\sin\psi\cos^3\theta}{8}\left(1 + z\theta' \Gamma_{56}\right)
\sqrt{z}\Gamma^5(P_- - P_+)e^{- \frac{\theta}{2}\Gamma_{56}}e^{- \frac{\theta}{2}\Gamma_{6,10}}\lambda_3\nonumber\\
& = -L_{DBI}^{-1}\frac{1}{z^5}\frac{\sin\psi\cos^3\theta}{8}\left(1 + z\theta' \Gamma_{56}\right)
\sqrt{z}\Gamma^5e^{+\frac{\theta}{2}\Gamma_{56}}e^{- \frac{\theta}{2}\Gamma_{6,10}}(P_- - P_+)\lambda_3\nonumber\\
& = -L_{DBI}^{-1}\frac{1}{z^5}\frac{\sin\psi\cos^3\theta}{8}\left(1 + z\theta' \Gamma_{56}\right)
\sqrt{z}\Gamma^5e^{+ \frac{\theta}{2}\Gamma_{56}}e^{- \frac{\phi}{2}\Gamma_{6,10}}(-e^{-\frac{\alpha}{2}\Gamma_{59}}e^{\frac{\psi}{2}\Gamma_{58}}e^{-\frac{\beta}{2}\Gamma_{59}}\eta_+ + \eta_-).
\end{align}
In order to satisfy the SUSY constraints we must have
\begin{align}
\left(1 - z\theta' \Gamma_{56}\right)  \propto e^{-\theta\Gamma_{56}},
\end{align}
or more concretely
\begin{align}
 z\theta'  = \tan\theta,
\end{align}
which is solved by
\begin{align}
\sin \theta(z) = c_1 z.
\end{align}
It is east to check that this also solves the EOM.
This solution starts with zero radius at the boundary, which grows to one at $z = 1/c_1$, and is not defined for larger $z$.
On the solution we have $L_{DBI} = \frac{\left(1-c_1^2 z^2\right) \sin \psi }{8 z^5}$, which also equals $\frac{\sin\psi\cos^2\theta}{8 z^5}$.


\subsection{D7 brane in PW}
For the D7 brane we take the worldvolume coordinates to be $x^\mu,r,\sigma_{1,2,3}$, and take the ansatz $\theta = \theta(r)$ and $\phi = \text{const.}$, and no gauge field on the D-brane.
To construct the kappa symmetry projector we need the pullback of $B$-field components
\begin{align}
[B_{r 1}] = \theta' B_{\theta 1} = -i a_1 \theta' \cos\phi ,\quad
[B_{2 3}] = B_{2 3} = - i a_2 \cos\phi.
\end{align}
Thus, we further impose $\phi = \pi/2$, so the $B$-field doesn't contribute to the kappa-projector, so we are left only with the $\gamma_{(8)}$ contribution, namely
\begin{align}
d^{8} \xi \Gamma = - e^{-\Phi} L_{\text{DBI}}^{-1} \gamma_{(8)}I
=- e^{-\Phi} L_{\text{DBI}}^{-1} \p_{a_1} X^{\mu_1}...\p_{a_8} X^{\mu_8}\gamma_{\mu_1...\mu_8}\frac{1}{8 !}d\xi^{a_8}\wedge ... \wedge d\xi^{a_1}I.
\end{align}
More explicitly we get
\begin{align}
\Gamma & = - e^{-\Phi} L_{\text{DBI}}^{-1} V_x^4\Gamma_{1234}\left(V_r \Gamma_5 + \theta' V_\theta \Gamma_6 \right)V_1 V_{23}^2\Gamma_{789}I\nonumber\\
& =  e^{-\Phi} L_{\text{DBI}}^{-1} V_x^4 V_r V_1 V_{23}^2 (\Pi^0_+-\Pi^0_-)\Gamma_{5789}\left(1  - \theta' \frac{V_\theta}{V_r} \Gamma_{56} \right)\nonumber\\
& =  - e^{-\Phi} L_{\text{DBI}}^{-1} V_x^4 V_r V_1 V_{23}^2 \Gamma_{11}\Gamma_{6,10}\left(1  - \theta' \frac{V_\theta}{V_r} \Gamma_{56} \right)I\nonumber\\
& =  e^{-\Phi} L_{\text{DBI}}^{-1} V_x^4 V_r V_1 V_{23}^2 \Gamma_{6,10}\left(1  - \theta' \frac{V_\theta}{V_r} \Gamma_{56} \right) I \nonumber\\
& =  - e^{-\Phi} L_{\text{DBI}}^{-1} V_x^4 V_r V_1 V_{23}^2 \left(P^0_+ - P^0_-\right) \left(1  - \theta' \frac{V_\theta}{V_r} \Gamma_{56} \right),
\end{align}
using the form of the Killing spinors we see that we need
\begin{align}
\cos\lambda \left(1  - \theta' \frac{V_\theta}{V_r} \Gamma_{56} \right)O(\lambda) = O(-\lambda),
\end{align}
or equivalently
\begin{align}
\cos\lambda \left(1  - \theta' \frac{V_\theta}{V_r} \Gamma_{56} \right)= O(-2\lambda) = \cos\lambda + \Gamma_{56}\sin\lambda,
\end{align}
so
\begin{align}
\theta' \frac{V_\theta}{V_r} = -\tan\lambda = - \sqrt{c A}\tan\theta.
\end{align}
Eliminating $r$, we get
\begin{align}
\frac{\sqrt{A}(1-c^2)}{\sqrt{c}}\frac{d\theta}{d c} = - \sqrt{c A}\tan\theta,
\end{align}
or
\begin{align}
\frac{d\theta}{d c} = - \frac{c}{1-c^2}\tan\theta.
\end{align}
The solution to this equation is given by
\begin{align}
\sin \theta(c) = k \sqrt{c^2 - 1} ,
\end{align}
where $k$ is a constant.
Interestingly, this is related to the $\AdS_5 \times \Sphere^5$ result by replacing $\sqrt{c^2 - 1} \to z$, which is exactly what we got for the D3-brane Wilson line analysis.
The deformed $\Sphere^3$ has unit radius at $c=1$, and shrinks to zero when $c = \sqrt{1+\frac{1}{k^2}}$.

From the gauge theory point of view one expects a phase transition, which in terms of the D-brane analysis means there should be other competing solutions. In the AdS case with $\Sphere^4$ slicing there are two topologically distinct solutions, one where the $\Sphere^3$ shrinks to a point for a finite value of the radial coordinate, so not all of AdS is covered (sort of an annulus), the other solution corresponds to the $\Sphere^3$ radius shrinking but staying finite as one approaches the center of AdS, so all of the AdS subspace is covered.
The first case is similar to the flat space slicing situation and the analysis above.

One can imagine relaxing our ansatz a bit to allow for more general solutions.
One way to do so is letting $\phi = \text{const.}$ be different then $\pi/2$. In this case it seems we should impose $K \epsilon_0 \sim \Gamma_{89}\epsilon_0$, although it seems it will give the same solution if such a solution exists.

\begin{align}
\Gamma & = -\frac{e^{\Phi}}{L_{\text{DBI}}}\Gamma_{1234}V_x^4\bigg(
\left(\Gamma_5 V_r  + \theta' V_\theta \Gamma_6\right)\Gamma_{789}V_1 V_{23}^2
+B_{r1}V_{23}^2\Gamma_{89} K\nonumber\\
&~~~+B_{23}\left(\Gamma_5 V_r  + \theta' V_\theta \Gamma_6\right)\Gamma_{7}V_1 K
+B_{r1}B_{23}
\bigg)I\nonumber\\
& = -\frac{e^{\Phi}}{L_{\text{DBI}}}\Gamma_{1234}V_x^4
\left(\left(\Gamma_5 V_r  + \theta' V_\theta \Gamma_6\right)\Gamma_{7}V_1
+B_{r1}K\right)\left(V_{23}^2\Gamma_{89}  + B_{23} K\right)
I\nonumber\\
& = -I\frac{e^{\Phi}}{L_{\text{DBI}}}\Gamma_{1234}V_x^4
\left(\left(\Gamma_5 V_r  + \theta' V_\theta \Gamma_6\right)\Gamma_{7}V_1
-B_{r1}K\right)\left(V_{23}^2\Gamma_{89}  - B_{23} K\right).
%& = G
%\left(\left(\Gamma_5 V_r  + \theta' V_\theta \Gamma_6\right)\Gamma_{7}V_1
%+B_{r1}K\right)\left(V_{23}^2\Gamma_{89}  + B_{23} K\right)
%\left(\Pi^0_- - \Pi^0_+\right),
\end{align}
%let's now impose the condition $K \eta = s i \Gamma_{89}\eta$ where $s = \pm 1$.
%Thus we have
%\begin{align}
%\Gamma\epsilon
%& = -I\frac{e^{\Phi}}{L_{\text{DBI}}}\Gamma_{1234}V_x^4
%\left(\left(\Gamma_5 V_r  + \theta' V_\theta \Gamma_6\right)\Gamma_{7}V_1 \left(V_{23}^2  - i s B_{23} \right)
%-i s B_{r1} \Gamma_{89} \left(V_{23}^2  + i s B_{23} \right)\right)\Gamma_{89}\epsilon,
%\end{align}
%Using the chirality condition $\Gamma_{1234}\epsilon = \Gamma_{56789,10}\epsilon$ we get
%\begin{align}
%\Gamma\epsilon
%& = -I\frac{e^{\Phi}}{L_{\text{DBI}}}V_x^4
%\left(-\left(V_r  + \theta' V_\theta \Gamma_{56}\right)V_1 \left(V_{23}^2  - i s B_{23} \right)
%-i s B_{r1} \Gamma_{5789} \left(V_{23}^2  + i s B_{23} \right)\right)\Gamma_{6,10}\epsilon,
%\end{align}
%
%
%
%
%\begin{align}
%\Gamma\epsilon
%& =  G
%\left(\left(\Gamma_5 V_r  + \theta' V_\theta \Gamma_6\right)\Gamma_{7}V_1
%-B_{r1} \Gamma_{89} \right)\Gamma_{57} O_1(-\lambda) O_3(\phi) O_2(\xi)\eta,
%\end{align}
%where $G \equiv \frac{e^{\Phi}}{L_{\text{DBI}}}V_x^4 \left(V_{23}^2  - s B_{23} \right) $.
%We can further write
%\begin{align}
%\Gamma\epsilon
%& =  G
%\left(-\left( V_r  + \theta' V_\theta \Gamma_{56}\right)V_1 O_1(-\lambda) O_3(\phi) O_2(\xi)\eta
%-B_{r1} \Gamma_{89} \Gamma_{57} O_1(-\lambda) O_3(\phi) O_2(\xi)\eta\right)\nonumber\\
%& =  G
%\left(-\left( V_r  + \theta' V_\theta \Gamma_{56}\right)V_1 O_1(-\lambda) O_3(\phi) O_2(\xi)\eta
%-B_{r1} O_1(\lambda) O_3(\phi) O_2(-\xi)\eta\right).
%\end{align}
%Now we multiply by the inverse, so
%\begin{align}
%\pm\eta
%& =  G
%\bigg(-V_1  \left( \left( V_r \cos\lambda  - \theta' V_\theta \sin\lambda\right) + (\theta' V_\theta \cos\lambda + V_r   \sin\lambda) \left(\cos\phi \Gamma_{56} + \sin\phi \cos\xi \Gamma_{5,10} + s \sin\phi \sin\xi \right) \right)\nonumber\\
%&-B_{r1} \left(\cos\xi + s \sin\xi \Gamma_{5,10} \right)\bigg)\eta.
%\end{align}
%Next we notice that $s$ must equal $\delta i$ with $\delta = \pm 1$, and remember that $i \eta = -\Gamma_{6,10}\eta$, so
%\begin{align}
%\pm\eta
%& =
%\bigg(-V_1  \left( \left( V_r \cos\lambda  - \theta' V_\theta \sin\lambda\right) + (\theta' V_\theta \cos\lambda + V_r   \sin\lambda) \left(\cos\phi \Gamma_{56} + \sin\phi \cos\xi \Gamma_{5,10} - \delta \Gamma_{6,10} \sin\phi \sin\xi \right) \right)\nonumber\\
%&-B_{r1} \left(\cos\xi - \delta  \sin\xi \Gamma_{56} \right)\bigg)\frac{e^{\Phi}}{L_{\text{DBI}}}V_x^4 \left(V_{23}^2  + \delta \Gamma_{6,10} B_{23} \right)\eta.
%\end{align}
%Thus, we need to solve the following equations
%\begin{align}
%&\Gamma_{56}\left(V_{23}^2\left(
%-V_1(\theta' V_\theta \cos\lambda + V_r   \sin\lambda) \cos\phi + \delta B_{r1}\sin\xi
%\right)
%-\delta B_{23}V_1(\theta' V_\theta \cos\lambda + V_r   \sin\lambda)\sin\phi \cos\xi
%\right)=0,\nonumber\\
%&\Gamma_{5,10}\left(V_{23}^2
%V_1(\theta' V_\theta \cos\lambda + V_r   \sin\lambda) \sin\phi \cos\xi
%-\delta B_{23}\left(
%-V_1(\theta' V_\theta \cos\lambda + V_r   \sin\lambda) \cos\phi + \delta B_{r1}\sin\xi
%\right)
%\right)=0,\nonumber\\
%&\Gamma_{6,10}\left(
%V_1 V_{23}^2(\theta' V_\theta \cos\lambda + V_r   \sin\lambda) \delta \sin\phi \sin\xi
%+\delta B_{23}\left(-V_1 \left( V_r \cos\lambda  - \theta' V_\theta \sin\lambda\right)
%-B_{r1} \cos\xi \right)
%\right)=0.
%%
%%\pm\eta
%%& =
%%\bigg(-V_1  \left( \left( V_r \cos\lambda  - \theta' V_\theta \sin\lambda\right) + (\theta' V_\theta \cos\lambda + V_r   \sin\lambda) \left(\cos\phi \Gamma_{56} + \sin\phi \cos\xi \Gamma_{5,10} - \delta \Gamma_{6,10} \sin\phi \sin\xi \right) \right)\nonumber\\
%%&-B_{r1} \left(\cos\xi - \delta  \sin\xi \Gamma_{56} \right)\bigg)\frac{e^{\Phi}}{L_{\text{DBI}}}V_x^4 \left(V_{23}^2  + \delta \Gamma_{6,10} B_{23} \right)\eta.
%\end{align}
However, it turns out that the only consistent solution is $\phi = 0$ as before.



Plugging the solution to the DBI action we get
\begin{align}
L_{DBI} = -\frac{A c \left(\left(c^2-1\right) M_f^2-1\right) \left(\left(c^2-1\right) M_f^2 (A c-1)+1\right)}{\left(c^2-1\right)^3}
\end{align}
up to a factor of $\frac{\sin^2\psi}{8}$ since we used the $\sigma_i$ coordinates.
In principle the Lagrangian should be integrated from $c = 1$ to $c = \sqrt{1 + \frac{1}{M_f^2}}$, however, this integral is divergent and we should used the method of holographic renormalization and add local counter terms to regulate the result.
We need the asymptotic form of $\theta(c)$ which is given by
\begin{align}
\theta(c) \simeq \sqrt{c-1}M_f\left(1 + \frac{M_f^2}{3}(c-1) + \frac{M_f^2(5+9M_f^2)}{30}(c-1)^2 + \mathcal{O}((c-1)^3)\right).
\end{align}

\subsection{$C_8$}
In order to evaluate the action and check the equations of motion, we also need the WZ term.
The only non-trivial contribution comes from the $C_8$ RR-form, where the other terms drop since $[B] = 0$ and $[d B] = 0$ at $\phi = \pi/2$.
We can find $C_8$ using the relation $\ast d C_0 = d C_8$, at $\phi = \pi/2$.
First we notice that although $C_0|_{\phi = \pi/2} = 0$, $d C_0|_{\phi = \pi/2} \neq 0$.
Generally we have $d C_0 = \p_\phi C_0 d\phi + \p_r C_0 d r + \p_\theta C_0 d\theta$, where the first term is not vanishing when $\phi = \pi/2$.
Taking the Hodge dual we get 
\begin{align}
\ast d C_0 = \sqrt{|g|}\left(-g^{\phi\phi}\p_\phi C_0 d r d\theta - g^{rr}\p_r C_0 d \theta d \phi + g^{\theta \theta}\p_\theta C_0 d r d\phi\right)d^4 x d V_3,
\end{align}
where $d^4 x = d X_1 \wedge ... \wedge dX_4$, and $dV_3 = \sigma_1 \wedge \sigma_2 \wedge \sigma_3$.    
Concentrating at the hyperplane $\phi = \pi/2$, and defining $\tilde F = -\sqrt{|g|}g^{\phi\phi}\p_\phi C_0$ and $C_8 = \left(C_\theta d\theta + C_r d r\right)d^4 x dV_3$, we find
\begin{align}
\tilde F = \p_r C_\theta - \p_\theta C_r.
\end{align}
Then, the pullback of $C_8$ will be given by 
\begin{align}
[C_8] = \left(C_\theta \theta'(r) + C_r \right)d^4 x d r dV_3.
\end{align}
This will give the following contribution to the equations of motion
\begin{align}
\theta'(r) \p_\theta C_\theta  + \p_\theta C_r - \p_r C_\theta = \theta'(r) \p_\theta C_\theta - \tilde F
=
-\frac{A^2 \cos^3 \theta \sin\theta}{(c^2-1)^2} = \p_\theta\left(\frac{A^2 \cos^4\theta}{4(c^2-1)^2}\right),
\end{align}
where we used the knowledge of what we should get on the RHS from the Euler-Lagrange equations for the DBI term.
However, it turns out that $\tilde F = RHS$, so $C_\theta = 0$, and we are only left with 
\begin{align}
[C_8]|_{\phi = \pi/2} = \frac{A^2 \cos^4\theta}{4(c^2-1)^2}d^4 dr dV_3.
\end{align}

%
%
%
%where $G\equiv -\frac{e^{\Phi}}{L_{\text{DBI}}}V_x^4$.
%Acting on $\epsilon = \Omega^{1/2} O_1(\lambda) O_3(\phi) O_2(\xi) \eta$ with $\eta = \Pi^0_- P^0_+ \epsilon_0$, we find
%\begin{align}
%\Gamma \epsilon
%& = G
%\left(\left(\Gamma_5 V_r  + \theta' V_\theta \Gamma_6\right)\Gamma_{7}V_1
%+B_{r1}K\right)\left(V_{23}^2\Gamma_{89}  + B_{23} K\right)
%\Omega^{1/2} O_1(\lambda) O_3(\phi) O_2(-\xi) \eta\nonumber\\
%\end{align}
%and the condition is
%\begin{align}
%\pm \eta
%& = G
%O_2(-\xi)
%O_3(-\phi)
%O_1(-\lambda)
%\left(\left(\Gamma_5 V_r  + \theta' V_\theta \Gamma_6\right)\Gamma_{7}V_1
%+B_{r1}K\right)\left(V_{23}^2\Gamma_{89}  + B_{23} K\right)
%O_1(\lambda) O_3(\phi) O_2(-\xi) \eta
%\end{align}
%\tiny\begin{align}
%\pm \eta
%& = G
%\left(\left[ \left(V_r \cos\lambda  - \theta' V_\theta \sin\lambda\right)\left(\cos\xi\Gamma_{5 7} - \sin\xi \Gamma_{5,10}K\right)
%  + \left(V_r \sin\lambda + \theta' V_\theta \cos\lambda\right)
%  \left(\cos\phi \cos\xi \Gamma_{6 7} - \cos\phi \sin\xi \Gamma_{6,10}K  - \sin\phi \Gamma_{7,10}\right)\right]V_1
%+B_{r1}\left(\cos\xi K - \sin\xi \Gamma_{7,10}\right)\right)
%\left(V_{23}^2\Gamma_{89}  + B_{23} K\right)\eta
%\end{align}









\appendix

\section{Notation}\label{app:notation}
Since the supergravity solution is quite involved and different conventions appear in the literature,
it is worthwhile to summarize the conventions we shall use in this paper.
\subsection{Background fields}
The metric will be denoted by $g_{\mu \nu}$ where $\mu,\nu = 1,...,10$, and we use the mostly minus convention.
We will use the indices $a,b=1,...,10$ to denote the metrics and other fields components in the non coordinate basis, related to the curved space by the use of the vielbeins $e_\mu^a$ and their inverse $E_a^\mu$.
Following \cite{Buchel:2000cn}, we denote by $\Phi$, $B_{(2)}$ and $C_{(n)}$ the dilaton, 2-form NSNS field and the RR potentials respectively.
We further define
\begin{align}\label{eq:defs1}
&C_{(0)} + i e^{-\Phi} = i\frac{B+1}{B-1},\nonumber\\
&A_{(2)} = C_{(2)} + i B_{(2)},\nonumber\\
&A_{(4)} = \frac{1}{4}\left(C_{(4)} + \frac{1}{2}B_{(2)}\wedge C_{(2)}\right),\nonumber\\
&F_{(3)} = d A_{(2)},\nonumber\\
&F_{(5)} = d A_{(4)} - \frac{1}{8} \Im (A_{(2)}\wedge F_{(3)}^*) = \frac{1}{4} d C_{(4)}.
\end{align}
We also have
\begin{align}\label{eq:defs2}
&\tilde F_{(1)} = d C_{(0)},\nonumber\\
&\tilde F_{(3)} = d C_{(2)} + C_{(0)} d B_{(2)},\nonumber\\
&\tilde F_{(5)} = d C_{(4)} + C_{(2)} \wedge d B_{(2)},\nonumber\\
&\tilde F_{(7)} = d C_{(6)} + C_{(4)} \wedge d B_{(2)},\nonumber\\
&\tilde F_{(9)} = d C_{(8)} + C_{(6)} \wedge d B_{(2)},\nonumber\\
\end{align}
and the duality relation
\begin{align}\label{eq:dualityconstraint}
\ast \tilde F_{(n+1)} =(-)^{n(n-1)/2} \tilde F_{(9-n)}.
\end{align}

Finally, the supergravity and Killing spinor equations involve the following filed combinations
\begin{align}\label{eq:defs3}
P_\mu & = f^2 \p_\mu B,\nonumber\\
Q_\mu & = f^2 \Im(B \p_\mu B^*),\nonumber\\
G_{\mu \nu \rho} & = f\left(F_{\mu \nu \rho} - B F_{\mu \nu \rho}^*\right).
\end{align}



\section{Killing Spinors in $\AdS_5 \times \Sphere^5$}
We use the gamma matrices and spin connection convention as in Schwarz (the chirality condition is given by $\Gamma^{11} \epsilon = \Gamma^1 ...\Gamma^{10}\epsilon = \epsilon$).
Our five form is given explicitly by
\begin{align}
F^{(5)}_{12345} = f,\quad
F^{(5)}_{6789,10} = f,
\end{align}
where these are the local frame indices and $f\pm 1$.
We need to solve
\begin{align}
& D_\mu \epsilon + \frac{i}{480} F_{\rho_1 \rho_2 \rho_3 \rho_4 \rho_5}\gamma^{\rho_1 \rho_2 \rho_3 \rho_4 \rho_5} \gamma_\mu \epsilon = 0
\end{align}
and we have $\frac{i}{480} F_{\rho_1 \rho_2 \rho_3 \rho_4 \rho_5}\gamma^{\rho_1 \rho_2 \rho_3 \rho_4 \rho_5}  = f\frac{i}{2} \Gamma^{12345}$, so
\begin{align}
& D_\mu \epsilon + f\frac{1}{2}\left(\Pi_+ - \Pi_-\right)\Gamma^5\gamma_\mu \epsilon = 0,
\end{align}
where $\Pi_\pm = \frac{1}{2}\left(1 \pm i\Gamma^{1234}\right)$.
\subsection{AdS Part}
The AdS spin connection is given by
\begin{align}
\omega_{\mu a b}\Gamma^{a b} = s 2 \gamma_\mu \Gamma^5, \mu=1,2,3,4,\quad
\omega_{\mu a b}\Gamma^{a b} =0,\quad \mu=5,
\end{align}
where $s=\pm 1$ (with $1$ corresponding to the standard sign as in Schwarz, although the tetrad postulate is consistent with $s=-1$).
This gives for $\mu=1,..,4$
\begin{align}
& \p_\mu \epsilon + s\frac{1}{2}\left(\Pi_+ + \Pi_-\right)\gamma_\mu \Gamma^5\epsilon  + f\frac{1}{2}\left(\Pi_+ - \Pi_-\right)\Gamma^5\gamma_\mu \epsilon = 0,
\end{align}
or
\begin{align}
& \p_\mu \epsilon + \frac{1}{2}\left((s-f)\Pi_+ + (s+f)\Pi_-\right)\gamma_\mu \Gamma^5\epsilon = 0.
\end{align}
The $z$ equation gives
\begin{align}
& \p_z \epsilon + f\frac{1}{2}\left(\Pi_+ - \Pi_-\right)\Gamma^5\gamma_\mu \epsilon = 0,
\end{align}
or equivalently
\begin{align}
& \p_z \epsilon_\pm \pm f\frac{1}{2z}\epsilon_\pm = 0.
\end{align}
Consistency of the $x^\mu$ and $z$ equations requires either
\begin{align}
& \p_\mu \epsilon_+ = 0,\quad
\p_\mu \epsilon_- - \gamma_\mu \Gamma^5\epsilon_+  = 0,\quad
\p_z \epsilon_\pm \mp\frac{1}{2z}\epsilon_\pm = 0,\quad s = f = -1,\nonumber\\
& \p_\mu \epsilon_- = 0,\quad
\p_\mu \epsilon_+ + \gamma_\mu \Gamma^5\epsilon_-  = 0,\quad
\p_z \epsilon_\pm \pm\frac{1}{2z}\epsilon_\pm = 0,\quad s = -f = -1.
\end{align}
These sets of equations are solved by
\begin{align}
&\epsilon_+ = \sqrt{z}\Gamma^5 \lambda_+,\quad
\epsilon_- = \frac{1}{\sqrt{z}}\lambda_- - x_\mu \gamma^\mu \sqrt{z}\lambda_+,\quad f = -1,\nonumber\\
&\epsilon_- = \sqrt{z}\Gamma^5 \lambda_-,\quad
\epsilon_+ = \frac{1}{\sqrt{z}}\lambda_+ + x_\mu \gamma^\mu \sqrt{z}\lambda_-,\quad f = +1.
\end{align}
\subsection{Sphere Part}
Using either $f=+1$ or $f=-1$ we have for the sphere part
\begin{align}
D_a \lambda_\pm-\frac{1}{2}\Gamma^5\Gamma_a \lambda_\pm = 0.
\end{align}
Next we construct the solution for various coordinate systems.
\subsubsection{spherical coordinates}
In spherical coordinates the metric is given by
\begin{align}
dS^2 = d\theta_1^2 + \sin^2\theta_1\left(d\theta_2^2 + \sin^2\theta_2\left(d\theta_3^2 + \sin^2\theta_3\left(d\theta_4^2 + \sin^2\theta_4 d\theta_5^2\right)\right)\right),
\end{align}
with $0\leq\theta_i\leq\pi$, except for $0\leq\theta_5\leq2\pi$.
For $\mu=6 $ we have $\omega_{\mu}{}^{ab}\Gamma_{ab} = 0$, so we need to solve
\begin{align}
\p_{\theta_1}\lambda + \frac{1}{2}\Gamma_{56}\lambda = 0,
\end{align}
which is solved by
\begin{align}
\lambda = e^{- \frac{\theta_1}{2}\Gamma_{56}}\lambda_2.
\end{align}
The next equation gives
\begin{align}
\p_{\theta_2}\lambda + \frac{1}{2}\cos\theta_1\Gamma_{67}\lambda+ \frac{1}{2}\sin\theta_1\Gamma_{57}\lambda
\p_{\theta_2}\lambda + \frac{1}{2}(\cos\theta_1\lambda-\sin\theta_1\Gamma_{56})\Gamma_{67}\lambda
=\p_{\theta_2}\lambda +\frac{1}{2}e^{-\Gamma_{56}}\Gamma_{67}\lambda = 0,
\end{align}
which becomes $\p_{\theta_2}\lambda_2 + \frac{1}{2}\Gamma_{67}\lambda_2 = 0$, so $\lambda_2 = e^{-\frac{\theta_2}{2}\Gamma_{67}}\lambda_3$.
Continuing with this analysis we should arrive at
\begin{align}
\lambda_{\pm} = e^{- \frac{\theta_1}{2}\Gamma_{56}}e^{- \frac{\theta_2}{2}\Gamma_{67}}e^{- \frac{\theta_3}{2}\Gamma_{78}}e^{- \frac{\theta_4}{2}\Gamma_{89}}e^{- \frac{\theta_5}{2}\Gamma_{9,10}}\eta_\pm,
\end{align}
where $\eta_\pm$ are constant spinors.


\subsubsection{Hopf coordinates}
Using Hopf coordinates the metric is given by
\begin{align}
dS^2 = d\theta^2 + \sin^2\theta d\phi^2 +\cos^2\theta d\Omega_3^2
\end{align}
where $d\Omega_3^2$ stands for the three sphere, $d\Omega_3^2 = \sum_{i=1}^{3}\sigma_i^2$.
In this case the $w_{\theta,a,b} = 0$, so as before
\begin{align}
\lambda = e^{- \frac{\theta}{2}\Gamma_{56}}\lambda_2.
\end{align}
Next we look at the $\phi$ equation,
\begin{align}
\p_{\theta_2}\lambda + \frac{1}{2}\cos\theta\Gamma_{610}\lambda+ \frac{1}{2}\sin\theta\Gamma_{5,10}\lambda
\p_{\theta_2}\lambda + \frac{1}{2}(\cos\theta\lambda-\sin\theta\Gamma_{56})\Gamma_{610}\lambda
=\p_{\theta_2}\lambda +\frac{1}{2}e^{-\Gamma_{56}}\Gamma_{610}\lambda = 0,
\end{align}
so $\lambda = e^{- \frac{\theta}{2}\Gamma_{56}}e^{- \frac{\phi}{2}\Gamma_{6,10}}\lambda_3$.
Next we have the three-sphere equations,
\begin{align}
D_a \lambda_\pm-\frac{1}{2}\Gamma^5\Gamma_a \lambda_\pm = 0.
\end{align}
The spin connections are given by
\begin{align}
\omega_{abc}\Gamma^{bc} = -\frac{2}{\cos\theta}\left(\sin\theta \Gamma_{6a} + \frac{1}{2}\epsilon_{abc}\Gamma^{bc}\right).
\end{align}
This boils down to
\begin{align}
\p_a\lambda_3 - \frac{1}{2}\left(-\Gamma_{5a} + \frac{1}{2}\epsilon_{abc}\Gamma^{bc}\right)\lambda_3
= \p_a\lambda_3 + \Gamma_{5a} \frac{1}{2}\left(1 + \Gamma_{5789} \right)\lambda_3 = 0,
\end{align}
where $\p_a$ is the dual of $\sigma_a$.
Notice that we can define the projectors $P_{\pm} = \frac{1}{2}\left(1 \pm \Gamma_{5789} \right)$, so that
\begin{align}
\p_a\lambda_3 + \Gamma_{5a} P_+ \lambda_3 = 0,
\end{align}
so we have
\begin{align}
\p_a\lambda_{3,+} + \Gamma_{5a} \lambda_{3,+} = 0,\quad
\p_a\lambda_{3,-} = 0.
\end{align}
We can parameterize the three-sphere is different ways, using the Euler angles ${x^{7,8,9} = \{\alpha,\beta,\psi\}}$ we have
\begin{align}
&\p_\alpha\lambda_{3,+} + \frac{1}{2}\Gamma_{59} \lambda_{3,+} = 0,\nonumber\\
&\p_\psi\lambda_{3,+} - \frac{1}{2}  \Gamma_{58} e^{\alpha\Gamma_{59}}\lambda_{3,+} = 0,\nonumber\\
&\p_\beta\lambda_{3,+} - \frac{1}{2}\left(\sin\psi\Gamma_{57} e^{\alpha\Gamma_{59}}-\cos\psi\Gamma_{59}\right)\lambda_{3,+} = 0,
\end{align}
which is solved by
\begin{align}
\lambda_{3,+} = e^{-\frac{\alpha}{2}\Gamma_{59}}e^{\frac{\psi}{2}\Gamma_{58}}e^{-\frac{\beta}{2}\Gamma_{59}}\eta_+,\quad
\lambda_{3,-} = \eta_-,
\end{align}
where $\eta_{\pm} = P_\pm\eta$, and $\eta$ is a constant spinor.
If one wants to use spherical coordinates for the three-sphere we can just quote the relevant piece from the above analysis using spherical coordinates for the five-sphere.



\bibliographystyle{nb}
\bibliography{refs}

\end{document}
