For a D-brane configuration there is an associated kappa symmetry projector, which, in Minkowski signature, is given by\cite{Skenderis:2002vf}:
\begin{align}
d^{p+1} \xi \, \Gamma = - \dfrac{ e^{\mathcal{F}}\wedge X|_{\text{Vol}}}{\sqrt{-\det  (g+\mathcal{F})}},
\end{align}
where
\begin{align}
X = \bigoplus_n \gamma_{(2n)} \mathcal{K}^n \mathcal{I},
\end{align}
and $|_{\text{Vol}}$ indicates projection to the volume form.
The operators $\mathcal{K}$, which is complex conjugation, and $\mathcal{I}$, act on a spinor $\psi$:
\begin{equation}
 \mathcal{K} \psi = \psi^* , \quad \mathcal{I} \psi = -i \psi.
\end{equation}
We also defined
\begin{align}
\gamma_{(n)} = \frac{1}{n !}d\xi^{i_n}\wedge ... \wedge d\xi^{i_1} P[\gamma]_{i_1...i_n},
\end{align}
built from the pullback of the gamma matrices in the curved target space. 

The kappa symmetry projector satisfies the traceless and the idempotent conditions.

