\section{Pilch-Warner geometry}\label{sec:PWB}

In this section we shall describe the Pilch-Warner solution to the type IIB supergravity equations.
We start by introducing the various background fields which are all non-trivial for this solution.
Our notation and conventions are summarized in appendix.
Afterwards, we introduce the Killing spinors for the background which were found first in \cite{Pilch:2003jg}.

\subsection{The boundary geometry}

The metric parametrization in \eqref{eq:PWmetric} is analogous to the horographic parametrization of the $AdS_5$ and Hopf parametrization for $S^5$ metric:
\begin{equation*}
ds^2=e^{2 r}\left(dx^\mu dx_\mu \right)-dr^2 -\left(d\theta^2+\phi^2 \sin ^2(\theta )+\cos ^2(\theta ) \left(\sigma_1^2+\sigma_2^2+\sigma_3^2\right)\right),
\end{equation*}
from where we can derive the metric in Poincare patch for the $AdS$ part,
\begin{equation*}
ds^2=\frac{dx^{\mu } dx_{\mu}-dz^2}{z^2},
\end{equation*}{}
using the change of coordinate:
\begin{equation*}
e^r=\frac{1}{z}.
\end{equation*}{}


The boundary of the Pilch-Warner geometry is located at $c = 1$.
Defining the coordinate $c = 1 + \frac{z^2}{2}$, and expanding around $z = 0$, we get the $AdS_5$ metric
\begin{align}
ds_E^2 =
\frac{dx^2 - dz^2}{z^2} + O(z^{0}).
\end{align}
Thus, close to the boundary, $z$ plays the role of the familiar radial AdS coordinate in Poincare coordinates.





