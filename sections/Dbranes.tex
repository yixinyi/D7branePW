\section{D-branes}

\subsection{Action}
The world-volume action of a single D$p$-brane consists of the Dirac-Born-Infeld (DBI) and the Wess-Zumino (WZ) or Chern Simons terms\footnote{An anti-brane corresponds to a sign change in front of the Wess-Zumino term, \cite{Kruczenski:2003be}.},
\cite{Ammon:2015wua}:
\begin{equation}
 S = S_{DBI} + S_{WZ},
\end{equation}
which are explicitly
\begin{align}
 S_{DBI} & = 
 -T_p \int_\mathcal{M} d^{p+1}\xi \, e^{-P[\Phi] } \sqrt{-\det (g+\mathcal{F})},\\
 S_{WZ} & =
 T_p\int _\mathcal{M} \sum_n e^{\mathcal{F}}\wedge P[C_{(n+1)}],
\end{align}
where $\xi$ are the coordinates for the worldvolume manifold $\mathcal{M}$, $g$ is the worldvolume metric (in string frame), $P[\cdot]$ denotes pullback from the target space, $\Phi$ is the dilaton, $C_{(n)}$ are the RR forms, and we defined
\begin{equation}
 \mathcal{F} \equiv \frac{1}{T_s} F + P[B_{(2)}], 
\end{equation}
with $F$ being the worldvolume field strength, and $B_{(2)}$ is the NSNS 2-form. Finally, the couplings, in terms of the string length $l_s$ and the string coupling constant $g_s$, are:
\begin{equation}
 T_{s} = \dfrac{1}{2\pi l_s^2}, \quad T_p = \dfrac{1}{g_s} T_s (2\pi l_s)^{1-p}.
\end{equation}


\subsection{Kappa symmetry projector}
For a D-brane configuration there is an associated kappa symmetry projector, which, in Minkowski signature, is given by\cite{Skenderis:2002vf}:
\begin{align}
d^{p+1} \xi \Gamma = - e^{-\Phi} L_{DBI}^{-1} e^{\mathcal{F}}\wedge X|_{\text{Vol}},
\end{align}
where
\begin{align}
L_{DBI}= e^{-\Phi} \sqrt{-\det  (g+\mathcal{F})}\quad;
\quad
X = \bigoplus_n \gamma_{(2n)} K^n I,
\end{align}
and $|_{\text{Vol}}$ indicates projection to the volume form.
The operators $K$ and $I$ act on a spinor $\psi$ in the following way:
\begin{equation}
 K \psi = \psi^* , \quad I \psi = -i \psi
\end{equation}


We also defined
\begin{align}
\gamma_{(n)} = \frac{1}{n !}d\xi^{i_n}\wedge ... \wedge d\xi^{i_1} \tilde{\gamma}_{i_1...i_n},
\end{align}
and
\begin{align}
\tilde{\gamma}_{i_1...i_n} = \p_{i_1} X^{\mu_1}...\p_{i_n} X^{\mu_n}\gamma_{\mu_1...\mu_n}.
\end{align}
The projector is traceless and idempotent $\Gamma^2 = 1$.




\subsection{Supersymmetric condition}
The condition for the D-brane configuration to be supersymmetric is that the kappa symmetry projector $\Gamma$ applied to the background Killing spinor $\epsilon$ fulfills
\footnote{The sign in front of the spinor is positive if mostly plus metric is used, for example, in \cite{Skenderis:2002vf}.}:
\begin{equation} \label{eq:susyCondition}
 \Gamma \epsilon = - \epsilon.
\end{equation}

If we are to impose the supersymmetric condition on an ansatz, it can provide us with first order differential equations, which are easier to solve than the standard second order equations of motion from the D-brane action. Here we will outline our strategy to solve the supersymmetric condition under certain conditions.

Let us consider the Killing spinor with the following structure:
\begin{equation}
\epsilon = \mathcal{O} \mathcal{P} \epsilon_0,
\end{equation}
where $\mathcal{O}$ is an invertible operator and $\mathcal{P}$ is a projector such that
\begin{equation}
    \mathcal{P} \epsilon_0 =  \epsilon_0,
\end{equation}
hence, there exists a complementary projector such that 
\begin{equation}
    \bar{\mathcal{P}} \epsilon_0 =  0.
\end{equation}

Then, the supersymmetric condition
\begin{equation}
 \Gamma \mathcal{O} \mathcal{P} \epsilon_0 = - \mathcal{O} \mathcal{P} \epsilon_0
\end{equation}
implies the following conditions
\begin{align}\label{eq:susyCondition0}
&\bar{\mathcal{P}} \mathcal{O}^{-1} \Gamma \mathcal{O} \mathcal{P}  = 0, \\
&\mathcal{P} \mathcal{O}^{-1} \Gamma \mathcal{O} \mathcal{P} = -\mathcal{P}.
\end{align}


We may find $n$ further projectors on the Killing spinor as necessary conditions for the above conditions to fulfill, for example, the kappa symmetry projector itself. Then, it means that the D-brane breaks $1/2^n$ copies of the background supersymmetry. 
























