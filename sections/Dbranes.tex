\section{D-branes}

\subsection{Action}
The world-volume action of a single D$p$-brane consists of the Dirac-Born-Infeld (DBI) and the Wess-Zumino (WZ) or Chern Simons terms, \cite{Ammon:2015wua}:

\begin{equation}
 S = S_{DBI} \pm S_{WZ},
\end{equation}
which are explicitly
\begin{align}
 S_{DBI} & = 
 -T_p \int_\mathcal{M} d^{p+1}\xi \, e^{-P[\Phi] } \sqrt{-\det (g+\mathcal{F})},\\
 S_{WZ} & =
 T_p\int _\mathcal{M} \sum_n e^{\mathcal{F}}\wedge P[C_{(n+1)}],
\end{align}
where $\xi$ are the coordinates for the worldvolume manifold $\mathcal{M}$, $g$ is the worldvolume metric (in string frame), $P[\cdot]$ denotes pullback from the target space, $\Phi$ is the dilaton, $C_{(n)}$ are the RR forms, and we defined
\begin{equation}
 \mathcal{F} := \frac{1}{T_s} F + P[B_{(2)}], 
\end{equation}
with $F$ being the worldvolume field strength, and $B_{(2)}$ is the NSNS 2-form. Finally, the couplings, in terms of the string length $l_s$ and the string coupling constant $g_s$, are:
\begin{equation}
 T_{s} = \dfrac{1}{2\pi l_s^2}, \quad T_p = \dfrac{1}{g_s} T_s (2\pi l_s)^{1-p}.
\end{equation}


\subsection{Kappa symmetry projector}
For a D-brane configuration there is an associated kappa symmetry projector, which, in Minkowski signature, is given by\cite{Skenderis:2002vf}:
\begin{align}
d^{p+1} \xi \Gamma = - e^{-\Phi} L_{DBI}^{-1} e^{\mathcal{F}}\wedge X|_{\text{Vol}},
\end{align}
where
\begin{align}
L_{DBI}= e^{-\Phi} \sqrt{-\det  (g+\mathcal{F})}\quad;
\quad
X = \bigoplus_n \gamma_{(2n)} K^n I,
\end{align}
and $|_{\text{Vol}}$ indicates projection to the volume form.
The operators $K$ and $I$ act on a spinor $\psi$ in the following way:
\begin{equation}
 K \psi = \psi^* , \quad I \psi = -i \psi
\end{equation}


We also defined
\begin{align}
\gamma_{(n)} = \frac{1}{n !}d\xi^{i_n}\wedge ... \wedge d\xi^{i_1} \tilde{\gamma}_{i_1...i_n},
\end{align}
and
\begin{align}
\tilde{\gamma}_{i_1...i_n} = \p_{i_1} X^{\mu_1}...\p_{i_n} X^{\mu_n}\gamma_{\mu_1...\mu_n}.
\end{align}
The projector is traceless and idempotent $\Gamma^2 = 1$.




\subsection{Supersymmetric condition}
The kappa symmetry projector $\Gamma$ applies to the background Killing spinor $\epsilon$. The condition for the D-brane configuration to be supersymmetric is that:
\begin{equation}
 \Gamma \epsilon = \epsilon.
\end{equation}
\comment{check the signs}


\subsection{Finding supersymmetric D-branes}
The susy condition has been used in literature to find BPS configurations, cite examples: Yamaguchi, ...

Here we outline our strategy to find our solutions.

Given an ansatz for a supersymmetric D-brane configuration, we can compute the kappa symmetry projector. 
Imposing the supersymmetric condition, we expect to infer first order conditions the ansatz must fulfill. 
This is in contrast to solving the second order equation of motion from the D-brane action.
\comment{The both methods should be equivalent...}


Let us assume the Killing spinor has the following structure:
\begin{equation}
\epsilon = \mathcal{O} \mathcal{P} \epsilon_0
\end{equation}
where $\mathcal{O}$ is an invertible operator and $\mathcal{P}$ is a projector such that
\begin{equation}
    \mathcal{P} \epsilon_0 =  \epsilon_0,
\end{equation}
hence, there exists a complementary projector such that 
\begin{equation}
    \bar{\mathcal{P}} \epsilon_0 =  0.
\end{equation}

The supersymmetric condition can be written as
\begin{align}
&\mathcal{P} \mathcal{O}^{-1} \Gamma \mathcal{O} \mathcal{P} = \mathbf{1} \\
&\bar{\mathcal{P}} \mathcal{O}^{-1} \Gamma \mathcal{O} \mathcal{P}  = \mathbf{0},
\end{align}
where the right hand side are the identity and null matrices.

In practice, the gamma matrices in the kappa symmetry projector are not independent, and one must employ the chirality condition and the projector conditions. \comment{check it}

We may find $n$ further projectors on the Killing spinor as necessary conditions for the above conditions to fulfill, then, it means the D-brane breaks $1/2^n$ copies more of the background supersymmetry.

\comment{Mention code?}
























