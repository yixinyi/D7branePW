\section{Integrate on-shell Lagrangian}\label{sec:integrate-action}

The on-shell action \eqref{eq:ActionAtSolution} can be integrated exactly. 
Since it is naturally expanded in powers of $L$, let us define:
\begin{equation}
 S = S_0 + S_2 + S_4.
\end{equation}
The solution of the respective indefinite integrals, up to constant terms, are:
\begin{align}
 I_0 &= - \int dc \,\left[ \frac{A(c)^2}{4 \left(c^2-1\right)^2}+\frac{c A(c)}{\left(c^2-1\right)^3} \right] \nonumber\\
     &=\frac{A(c)-\left(A(c)^2+2\right) c+3 A(c) c^2}{4 \left(c^2-1\right)^2},\label{eq:I0}\\
    \quad\nonumber\\
%     
I_2 &= - \int dc \, \frac{L^2 A(c) \left(\left(c^2+1\right) A(c)-4 c\right)}{2 \left(c^2-1\right)^2}\nonumber\\
    &=-L^2 \left[\frac{\left(c^3+3 c-4\right) (A(c)-c)^2}{6 \left(c^2-1\right)^2} 
      +\frac{c \left(5-2 c^2\right)}{6 \left(c^2-1\right)} + \, \frac{1}{3} \log^2\left(\frac{c+1}{2}\right)\right.\nonumber\\   
 & \quad \left. +\frac{1}{6} \left(2 A(c)+ c\right)
   + \frac{2}{3}  \, \text{Re} \, \text{Li}_2\left(\frac{c+1}{2}\right)\right],\label{eq:I0}\\
   \quad\nonumber\\
%    
 I_4 &=  \int dc\, \frac{L^4 A(c) \left(3 c^2 A(c)+A(c)-4 c\right)}{4 \left(c^2-1\right)} \nonumber\\
 &= L^4 \left[\frac{\left(9 c^4-7 c^2-2\right) (A(c)-c)}{30 \left(c^2-1\right)}+\frac{\left(9 c^5-10 c^3-15 c+16\right) (A(c)-c)^2}{60 \left(c^2-1\right)^2}\right.\nonumber\\
 & \quad \left. + \frac{3 \, c^3}{20}-\frac{c}{15} -\frac{2}{15} \log ^2\left(\frac{c+1}{2}\right) - \frac{4}{15}\, \text{Re} \, \text{Li}_2\left(\frac{c+1}{2}\right) \right]. \label{eq:I0}
\end{align}

Next, we are going to expand these integrals at different limits. The following expansions for small $\epsilon$ will be useful:
\begin{align}
& \text{Li}_2\left(1 + \epsilon \right) \approx  \zeta(2) + \frac{\pi^2}{4} \epsilon + \ldots,\\
& A(1+\epsilon^2/2)\approx 1+\frac{1}{2} \epsilon^2 \left(2 \log \left(\frac{\epsilon}{2}\right)+1\right) 
+ \frac{1}{8} \epsilon^4 \left(2 \log \left(\frac{\epsilon}{2}\right)-1\right)+ \ldots \label{eq:expandA}
\end{align}

\subsection{UV limit}\label{sec:UVlimit}
Let us evaluate the integrals at the lower bound $c = 1 + \epsilon^2/2$, and expand in small $\epsilon$. The leading order terms are:
\begin{align}
 I_0 &\approx \frac{1}{4 \epsilon ^4} +\frac{1+\log \left(\epsilon/2\right)}{2 \epsilon ^2}-\frac{ \log ^2\left(\epsilon/2\right)}{4}+\frac{\log (\epsilon )}{8} + \frac{3}{64} + O(\epsilon^2),\\
%     
I_2 & \approx -\frac{L^2}{2 \epsilon^2}-\frac{7 L^2}{24}-\frac{\pi ^2 L^2}{9} + O(\epsilon^2),\\
%    
 I_4 & \approx \frac{L^4}{12}-\frac{2 \pi ^2 L^4}{45} + O(\epsilon^2). 
\end{align}


\subsection{IR limit and large L} \label{sec:IRlimit}
Let us evaluate the integrals at $c_\text{max}$, and expand in large $L$. That means 
\begin{equation}
 c_\text{max} \approx 1 + \frac{1}{2 L^2} - \frac{1}{8 L^4} + O(L^{-6}).
\end{equation}
The leading order terms for the integrals are:
\begin{align}
 I_0 &\approx \frac{L^4}{4}+\frac{1}{2} \left(\frac{5}{4} - \log(2 L)\right) L^2 + O(\log L),\\
%     
I_2 & \approx -\frac{L^4}{2} -\left(\frac{5}{12}+\frac{\pi^2}{9}\right) L^2 + O(\log L),\\
%    
 I_4 & \approx \left(\frac{1}{12}-\frac{2 \pi ^2}{45}\right) L^4+ \frac{1}{120} \left(23 -2 \pi^2 - 44 \log(2 L) \right)L^2+ O(\log L). 
\end{align}
Their sum is:
\begin{equation}\label{eq:IatUV}
 I_0+I_2+I_4 \approx -\left(\frac{1}{6}+\frac{2 \pi ^2}{45}\right) L^4
             +\left(\frac{2}{5}-\frac{23 \pi ^2}{180}-\frac{ 13}{15} \log(2 L)\right)L^2 + O(\log L).
\end{equation}


