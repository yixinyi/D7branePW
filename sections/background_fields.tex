The Pilch-Warner solution has non-trivial background fields. Following the conventions of \cite{Buchel:2000cn} and \cite{Pilch:2003jg}. The dilaton $\Phi$ and axion $C_{(0)}$ fields are given by:
\begin{equation}
 e\,^{-\Phi }-i C_{(0)} =\frac{1+\mathcal{B}}{1-\mathcal{B}}\,,
 \qquad
 \mathcal{B}=\,e\,^{2i\phi }\,\frac{\sqrt{cX_1}-\sqrt{X_2}}{\sqrt{cX_1}+\sqrt{X_2}}.
\end{equation}
The 2-form potential that is the linear combination of the RR and the NSNS 2-form potentials, $A_{(2)}=C_{(2)}+i B_{(2)}$, is:
\begin{equation}
 A_{(2)} = e^{i \phi}\left(i a_1 \, d\theta \wedge \sigma_1 + i a_2 \, \sigma_2 \wedge \sigma_3 + a_3\, \sigma_1 \wedge d\phi\right),
\end{equation}
with the real functions\footnote{Notice that we factored out the imaginary $i$, in contrast to \cite{Pilch:2003jg}.}:
\begin{align}
a_1(c,\theta) = & - \frac{\sqrt{c^2-1}}{c}\cos\theta,\nonumber\\
a_2(c,\theta) =  & A  \frac{\sqrt{c^2-1}}{X_1}\sin \theta \cos^2 \theta,\nonumber\\
a_3(c,\theta) =  &  -\frac{\sqrt{c^2-1}}{X_2}\sin \theta \cos^2 \theta.
\end{align}
And the self-dual 5-form field strength $\tilde{F}_{(5)}$ is given by:
\begin{equation}
\tilde{F}_{(5)} = \mathcal{F} + *\mathcal{F},
\qquad
\mathcal{F} = 4 d{x^0} \wedge d{x^1} \wedge d{x^2} \wedge d{x^3} \wedge d\omega(c,\theta),
\end{equation}
where 
\begin{equation}
\omega(c,\theta) = \frac{{A{\kern 1pt} {X_1}}}{{4{{\left( {{c^2} - 1} \right)}^2}}}.
\end{equation}

Using the following definitions for the field strengths:
\begin{align}\label{eq:defs2}
&\tilde F_{(1)} = d C_{(0)},\nonumber\\
&\tilde F_{(3)} = d C_{(2)} + C_{(0)} d B_{(2)},\nonumber\\
&\tilde F_{(5)} = d C_{(4)} + C_{(2)} \wedge d B_{(2)},\nonumber\\
&\tilde F_{(7)} = d C_{(6)} + C_{(4)} \wedge d B_{(2)},\nonumber\\
&\tilde F_{(9)} = d C_{(8)} + C_{(6)} \wedge d B_{(2)},\nonumber\\
\end{align}
and the Hodge duality relation 
\begin{align}\label{eq:dualityconstraint}
\ast \tilde F_{(n+1)} =(-)^{n(n-1)/2} \tilde F_{(9-n)},
\end{align}
we could compute all the RR-forms $C_{(n)}$ and the NSNS-form $B_{(2)}$, up to an exact form. 

We would like to comment on the Hodge star operation, which is defined as:
\begin{equation}
 \ast (dx^{i_1} \wedge \cdots \wedge dx^{i_k} ) = 
 \dfrac{\sqrt{|\det g|}}{(n-k)!} g^{i_1 j_1} \cdots g^{i_k j_k} \epsilon_{j_1 \ldots j_n} (dx^{j_{k+1}} \wedge \cdots \wedge dx^{j_n} ),
\end{equation}
where $g$ is the metric, and the Levi-Civita symbol satisfying $\epsilon_{1 \ldots n}=1$.
Notice the metric to be used here is the metric in the string frame. This is important since the dilaton term is non-trivial in the Pilch-Warner background, unlike in $AdS_5 \times S^5$. 

Although we do not need all the potentials for our problem, we list the explicit solutions below:

\begin{align*}
e^{\Phi} & =
\frac{c \sin^2\phi \, X_1+\cos^2\phi \,X_2}{\sqrt{c X_1 X_2}}\\
%
C_{(0)} & = 
-\frac{\sin(2 \phi ) (c X_1-X_2)}{2  (c \sin^2\phi \, X_1+\cos^2\phi \,X_2)}\\
%
C_{(2)} & = 
- a_2 \sin\phi \,
\sigma_2\wedge \sigma_3
+ a_3 \cos\phi \,
\sigma_1\wedge d\phi
+ a_1 \sin\phi \,
\sigma_1\wedge d\theta\\
% 
B_{(2)} & =
a_2 \cos\phi \,
\sigma_2\wedge \sigma_3
+ a_3 \sin\phi \,
\sigma_1\wedge d\phi
- a_1 \cos\phi \,
\sigma_1\wedge d\theta\\
% 
C_{(4)} & =
4 \, \omega \,
dx_0\wedge dx_1\wedge dx_2\wedge dx_3 
+ \sin\phi \cos\phi \, a_1 a_2 \,
d\theta\wedge \sigma_1\wedge \sigma_2\wedge \sigma_3\\
&
-\left(\cos ^2\phi \, a_2 a_3 +\frac{c \cos^4\theta}{X_2}\right) 
\sigma_1\wedge \sigma_2\wedge \sigma_3 \wedge d\phi \\
%
C_{(6)} &= - C_{(4)} \wedge B_{(2)}\\
&+\frac{(c A-2) A \sin\theta \cos^2\theta \cos\phi}{2 \left(c^2-1\right)^{5/2}} 
\sigma_1\wedge dc\wedge dx_0\wedge dx_1\wedge dx_2\wedge dx_3\\
&
+\frac{A^2 \cos^3\theta \cos\phi}{2 (c^2-1)^{3/2}}
\sigma_1\wedge dx_0 \wedge dx_1\wedge dx_2\wedge dx_3\wedge d\theta\\
&
-\frac{A^2 \sin\theta \cos^2\theta \sin\phi}{2 (c^2-1)^{3/2}}
\sigma_1\wedge dx_0 \wedge dx_1\wedge dx_2\wedge dx_3\wedge d\phi\\
%
C_{(8)} &= - C_{(6)} \wedge B_{(2)}\\
&+ \left\{ \frac{A \sin^2\theta \cos^4\theta \cos(2\phi) \left[ ((c^2+3) A-4 c) \cos^2\theta-2 A \right]}{4 (c^2-1)^2 X_2^2} \right.\\
&+ \left.\frac{A ((c^2-1) A + 4 c) \cos^4\theta}{8 c^2 (c^2-1)^2} \right\}
\sigma_1 \wedge \sigma_2\wedge \sigma_3\wedge dc\wedge dx_0\wedge dx_1\wedge dx_2\wedge dx_3\\
& -\frac{A^2 \sin^3(2 \theta ) \cos(2 \phi ) \left(2 c A + A^2 \sin^2\theta +c^2 \cos^2\theta\right)}{16 c (c^2-1) X_2^2}\\
&\sigma_1\wedge \sigma_2\wedge \sigma_3\wedge dx_0\wedge dx_1\wedge dx_2\wedge dx_3\wedge d\theta
\end{align*}


