
The Killing spinor solution $\epsilon$, originally derived in \cite{Pilch:2003jg} and we follow the appendix of \cite{Chen-Lin:2015xlh}, has the following expression:
\begin{align}
\epsilon &= e^{i\phi/2}\Omega^{1/2} O_\alpha O_\beta \eta
\end{align}
where we defined
\begin{equation}
\eta = \Pi_{-}^0 \mathcal{P}_+^{0} \epsilon_0
\end{equation}
and the constant spinor $\epsilon_0$ satisfies
\begin{equation}
 \Pi^0_- \mathcal{P}_+^{0} \epsilon_0 = \epsilon_0.
\end{equation}

The projectors are
\begin{align}
\Pi_\pm^0 = 
\frac{1}{2}\left(1 \pm i\Gamma^{1234}\right), 
\quad
\mathcal{P}_{\pm}^{0} =
\dfrac{1}{2} \left(1\pm i\Gamma^{6,10}\right)
\end{align}
and the operators are
\begin{align}
O_\alpha = \cos\frac{\alpha}{2} + \sin\frac{\alpha}{2}\Gamma^{5 6},
\quad
O_\beta = \cos\frac{\beta}{2} + \sin\frac{\beta}{2}\Gamma^{7, 10}\mathcal{K}
\end{align}
where $\mathcal{K}$ is complex conjugation, and the angles are defined as
\begin{align}
\cos\beta = \frac{X_1}{c X_2}, \quad 
&\quad
\sin\beta = -\sqrt{\frac{c^2 - 1}{c X_2}}\cos\theta\\
%
\cos\alpha = \frac{\cos\theta}{\sqrt{X_1}}, 
&\quad
\sin\alpha = \frac{\sqrt{c}A^{1/2}\sin\theta}{\sqrt{X_1}},
\end{align}
with $\Omega, X_{1,2}, A$ functions of $c$ and $\theta$, as defined in section \ref{sec:metric}. 


The gamma matrices are in real representations. It is convenient to write the $e^{i \frac{\phi}{2}}$ factor in real representation too. 
That is achieved by using $\mathcal{P}_-^0 \eta = 0 $, hence
\begin{align}
 e^{i \frac{\phi}{2}} \eta 
    = e^{-\frac{\phi}{2}\Gamma^{6,10}} \eta
    = O_\phi \eta,
\end{align}
where 
\begin{align}
 O_\phi = \cos\frac{\phi}{2} - \sin\frac{\phi}{2}\Gamma^{6, 10}.
\end{align}

The Killing spinor can be rewritten as:
\begin{equation}\label{eq:KillingSpinor}
 \epsilon = \Omega^{1/2} O_\alpha O_\phi O_\beta \Pi_{-}^0 \mathcal{P}_+^{0} \epsilon_0.
\end{equation}




We number the coordinates\footnote{We abuse the notation by calling $\sigma_i$ coordinates of the deformed 3-sphere.} in the following order:
\begin{equation}
 \{x_1, x_2, x_3, x_4, c, \theta, \sigma_1, \sigma_2, \sigma_3, \phi\}. 
\end{equation}

Note also that, since we are using $c$ instead of $r$ as coordinate, there is a flip of sign for $\Gamma_5$, seen e.g. in $O$, in contrast to \cite{Pilch:2003jg}.

\comment{Use subindex for gamma matrices}
