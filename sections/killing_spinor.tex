
The Killing spinor solution $\epsilon$, originally derived in \cite{Pilch:2003jg} and we follow the appendix of \cite{Chen-Lin:2015xlh}, has the following expression:
\begin{align}
\epsilon &= \Omega^{1/2} \exp{\left(\frac{\phi}{2}\, i \right)} \exp{\left(\frac{\alpha}{2}\Gamma^{56} \right)} \exp{\left(\frac{\beta}{2}\Gamma^{7, 10} \mathcal{K} \right)} \eta
\end{align}
where we defined
\begin{equation}
\eta = \Pi_{-} \mathcal{P}_+ \epsilon_0
\end{equation}
and the constant spinor $\epsilon_0$ satisfies
\begin{equation}
 \Pi_- \mathcal{P}_+ \epsilon_0 = \epsilon_0.
\end{equation}

The projectors are
\begin{align}
\Pi_\pm = 
\frac{1}{2}\left(1 \pm i\Gamma^{1234}\right), 
\quad
\mathcal{P}_{\pm} =
\dfrac{1}{2} \left(1\pm i\Gamma^{6,10}\right),
\end{align}
and they commute with each other: $[\Pi_\pm, \mathcal{P}_{\pm}]=0$.

The exponentials can be written in terms of cosines and sines:
\begin{align}
\exp{\left(\frac{\alpha}{2}\Gamma^{56} \right)} = \cos\frac{\alpha}{2} + \sin\frac{\alpha}{2}\Gamma^{5 6},
\quad
\exp{\left(\frac{\beta}{2}\Gamma^{7, 10} \mathcal{K} \right)} = \cos\frac{\beta}{2} + \sin\frac{\beta}{2}\Gamma^{7, 10}\mathcal{K}
\end{align}
where $\mathcal{K}$ is complex conjugation, and the angles are defined as
\begin{align}
\cos\beta = \sqrt{\frac{X_1}{c X_2}}, \quad 
&\quad
\sin\beta = -\sqrt{\frac{c^2 - 1}{c X_2}}\cos\theta\\
%
\cos\alpha = \frac{\cos\theta}{\sqrt{X_1}}, 
&\quad
\sin\alpha = \sqrt{\frac{c A}{X_1}}\sin\theta,
\end{align}
where $\Omega, X_{1,2}, A$ are functions of $c$ and $\theta$ defined in section \ref{sec:metric}.


The gamma matrices are in real representations. It is convenient to write the $e^{i \frac{\phi}{2}}$ factor in real representation too. 
That is achieved by using $\mathcal{P}_- \eta = 0 $, which leads to
\begin{align}
 \exp{\left(\frac{\phi}{2}\, i \right)} \eta 
    = \exp{\left(-\frac{\phi}{2}\Gamma^{6, 10} \right)} \eta. 
\end{align}

After some straightforward manipulations, the Killing spinor can be rewritten as:
\begin{equation}\label{eq:KillingSpinor}
\boxed{ \epsilon =  \Omega^{1/2} \exp{\left(\frac{\alpha}{2}\Gamma^{56} \right)} \exp{\left(-\frac{\phi}{2}\, \Gamma^{6,10} \right)} \exp{\left(\frac{\beta}{2}\Gamma^{7, 10} \mathcal{K} \right)} \Pi_{-} \mathcal{P}_+ \epsilon_0}.
\end{equation}

\comment{Moved this part?}
We number the coordinates\footnote{We abuse the notation by calling $\sigma_i$ coordinates of the deformed 3-sphere.} in the following order:
\begin{equation}
 \{x_1, x_2, x_3, x_4, c, \theta, \sigma_1, \sigma_2, \sigma_3, \phi\}. 
\end{equation}


\comment{Use subindex for gamma matrices? Specify signature}
