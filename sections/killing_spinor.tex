
The Killing spinor solution $\epsilon$, derived in \cite{Pilch:2003jg}, has the following expression:
\begin{align}
\epsilon = \frac{e^{i\phi/2}\Omega^{1/2}}{\cos\frac{\beta}{2}}O(\alpha)
\Pi_-(\beta) \mathcal{P}_+^{0} \Pi^0_-\epsilon_0, \quad 
\mathcal{P}_+^{0} \Pi^0_-\epsilon_0 = \epsilon_0.
\end{align}
where the projectors are
\begin{align}
\Pi_\pm (\beta) &= 
\frac{1}{2}\left( 1 \pm i\Gamma^{1234} \left(\cos\beta - \sin \beta \Gamma^{7,10} * \right)\right)\\
\Pi_\pm^0 &= 
\frac{1}{2}\left(1 \pm i\Gamma^{1234}\right)\\
\mathcal{P}_{\pm}^{0} &=
\dfrac{1}{2} \left(1\pm i\Gamma^{6,10}\right)
\end{align}
and 
\begin{align}
O(\alpha)  &= 
\cos\frac{\alpha}{2} - \sin\frac{\alpha}{2}\Gamma^{5 6}\\
\cos\beta \equiv \frac{X_1}{c X_2}, 
&\quad
\sin\beta \equiv -\sqrt{\frac{c^2 - 1}{c X_2}}\cos\theta\\
\cos\alpha \equiv \frac{\cos\theta}{\sqrt{X_1}}, &\quad
\sin\alpha \equiv \frac{\sqrt{c}A^{1/2}\sin\theta}{\sqrt{X_1}}.
\end{align}
and $\Omega, X_{1,2}, A$ are defined in section \ref{sec:metric}.


\comment{Move this to the main text?}
In the  $\theta\to\pi/2, \phi \to 0$ limit, 
$\beta = 0$ and $\alpha = \pi/2$ and $\Pi_-(\beta=0) = \Pi_-^0$, hence we find
\begin{align}\label{eq: KillingSpinorClassic}
\epsilon(\theta=\pi/2,\phi=0) = \frac{\sqrt{2}c^{1/8}A^{1/4}}{(c^2-1)^{1/4}}
\frac{1}{2}(1-\Gamma^{56})\mathcal{P}_+^{0} \Pi^0_-\epsilon_0.
\end{align}
Notice that $\frac{1}{2}(1-\Gamma^{56})$ is not a projector. 
