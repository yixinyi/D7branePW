\section{Introduction}

The holographic principle promises to be a useful framework to tackle strongly coupled gauge theories by means of weakly coupled string theories. Its best known instance is the AdS/CFT duality, conceived by Maldacena in his seminar work \cite{Maldacena:1997re}. In particular, the duality involves a strongly coupled $SU(N)$ $\mathcal{N}=4$ SYM, which is a CFT, and the supergravity in the $AdS_5 \times S^5$ background. This is a very well understood case by now, and despite not describing any real world systems, AdS/CFT is being used as a theoretical laboratory to explore dynamics of real strongly coupled systems.

The most famous example is undoubtely Quantum Chromodynamics (QCD). To compare with it, we must first add flavors to $\mathcal{N}=4$ theory,  or in other words, quarks. This means an additional hypermultiplet sector with fields in the fundamental representation of the gauge group $SU(N)$. The equivalent in the holographic picture is the insertion of $N_f$ probe D7-branes embedding in $AdS_5 \times S^5$, \cite{Karch:2002sh}. The quarks arise from strings stretched between the D7 and the $N$ coincindent D3-branes that generate the supergravity background. When the D7-branes and the D3-branes are separated, the quarks become massive. Significant work has been done in this direction, and we refer readers to the following reviews \cite{CasalderreySolana:2011us, Erdmenger:2007cm}.

In this paper, we study the flavor dynamics in a less symmetric theory called $\mathcal{N}=2^*$ SYM, via the holographic correspondence. This theory results from breaking the conformal invariance of $\mathcal{N}=4$ SYM by adding mass to its adjoint hypermultiplet. Consequently, the supersymmetries are also halved. Its holographic dual is known too, namely the Pilch-Warner supergravity \cite{Pilch:2000ue, Pilch:2003jg}. This geometry consists of a product space of a warped $AdS_5$ and a squashed $S^5$, and it is asymptotically $AdS_5 \times S^5$ near its boundary. This instance of the holographic duality is a non-conformal extension of AdS/CFT, and has been extensively studied and tested in the last few years \cite{Buchel:2013id, Chen-Lin:2015xlh, Chen-Lin:2017pay, Russo:2019lgq}. It is therefore a natural step to extend the flavor sector in $\mathcal{N}=2^*$ theory, and in fact, it has been done in \cite{Albash:2011nw}. 

We provide a simple closed-form solution, by solving the supersymmetric condition imposed on the probe D7-brane embedding. What \cite{Albash:2011nw} found was a perturbative near-boundary solution, with a surprising logarithmic term. We will show that the latter must be absent, because that the authors wrongly concluded the absence of the $C_{(8)}$ potential term in the D-brane Lagragian. 

The present paper starts, in section \ref{sec:Dbranes}, by reviewing D-branes and outlining the strategy we will use to find supersymmetric embeddings. Then, in section \ref{sec:D7brane}, we deal with our D7-brane in detail, and find the right configuration. We will conclude by describing the implications of our results. The Pilch-Warner background is summarized in the appendix, including explicit forms of the Ramond-Ramond (RR) potentials that we computed for completeness. 




