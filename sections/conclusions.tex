\section{Conclusion}

In this paper, we found an exact supersymmetric D7-brane embedding in the Pilch-Warner geometry by solving the supersymmetric condition. In the field theory side, this setting corresponds to having a quark sector in the $\mathcal{N}=2^*$ SYM at zero temperature. Our D7-brane configuration reduces to the known solution in the AdS/CFT case \cite{Karch:2005ms}, in the near-boundary limit. 

We demonstrated that the pullback of $C_{(8)}$ for values of $\phi$ that are odd fractions of $\pi$ is non-zero. This is important since previous papers such as \cite{Albash:2011nw} and \cite{Evans:2005ti} argued that the pullback of $C_{(8)}$ is vanishing while using the Einstein metric frame. In particular, the embedding in \cite{Albash:2011nw} that differs from ours only in the WZ term, has a logarithmic term that leads to a non-vanishing quark condensate, violating therefore supersymmetry. Our supersymmetric condition method does not require the D-brane Lagrangian, so that the fulfilment of the equation of motion at the solution provides a strong proof that the WZ term is there, besides our explicit computations of the fluxes. We can conclude also that the string frame is the right metric frame to use in the D-brane analysis, instead of the Einstein frame. 

As possible future work, the fluctuations around our D-brane, corresponding to the meson spectrum on the field theory side, could be studied. It would also be interesting to consider the scenario of a non-vanishing gauge field on the deformed sphere. For example, in $AdS_5 \times S^5$, \cite{Kruczenski:2003be} studied mesons that carry angular momentum. This scenario has also been studied by \cite{Karch:2015vra} for the global $AdS_5 \times S^5$, where different topological inequivalent solutions were found. These same authors in \cite{Karch:2015kfa} compared their results directly with the field theory results computed via the supersymmetric localization technique \cite{Pestun:2007rz} for $\mathcal{N}=4$ SYM on $S^4$. In principle, a similar precision test could be done for $\mathcal{N}=2^*$ SYM too, and perhaps we could learn more about the nature of the interersting phase transitions observed in this theory \cite{Russo:2013qaa, Zarembo:2014ooa, Chen:2014vka}. One could also review the thermal case studied in \cite{Albash:2011dq}, for which the geometry was derived in \cite{Buchel:2003ah}. Another avenue would be the study of various probe branes, for example, \cite{Faedo:2019jlp} recently explored the two D7-branes case for $AdS_5 \times S^5$.


