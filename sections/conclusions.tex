\section{Conclusion}

In this paper, we found an exact supersymmetric D7-brane embedding in the Pilch-Warner geometry by solving the supersymmetric condition. In the field theory side, this setting corresponds to having a quark sector in the $\mathcal{N}=2^*$ SYM at zero temperature. Our D7-brane configuration reduces to the known solution in the AdS/CFT case \cite{Karch:2005ms}, in the near boundary limit. Our solution also resums the asymptotic series of \cite{Albash:2011nw} and simplify some of their analysis, for example, the quark condensate is zero, a result that is compatible with supersymmetry. 

We also demonstrated that the pullback of $C_{(8)}$ for values of $\phi$ that are odd fractions of $\pi$ is non-zero, in contrast to authors in \cite{Albash:2011nw} and \cite{Evans:2005ti}. Our supersymmetric condition method does not require the D-brane Lagrangian, hence the fulfilment of the equation of motion at the solution provides a strong proof that the WZ term is there, besides our explicit computations of the fluxes. We can conclude also that the string frame is the right metric frame to use in the D-brane analysis, instead of the Einstein frame. 

As a possible future work, we can study the fluctuations around our D-brane, hence the meson spectrum in the field theory side. 
It is also interesting to consider non-vanishing gauge fields on the deformed sphere, as done in \cite{Kruczenski:2003be} for the $AdS_5 \times S^5$ case, where their mesons carry angular momentum. One can also review the thermal case studied in \cite{Albash:2011dq}, where the geometry was derived in \cite{Buchel:2003ah}. Many probe branes is another direction, recently explored by \cite{Faedo:2019jlp} for $AdS_5 \times S^5$ with two branes.


