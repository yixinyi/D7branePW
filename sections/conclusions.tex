\section{Conclusion}

In this paper, we found an exact half-BPS D7-brane embedding in the Pilch-Warner background, by solving the supersymmetric condition. In the field theory side, this setting corresponds to adding a quark sector in the $\mathcal{N}=2^*$ SYM at zero temperature. Because the embedding breaks half of the background supersymmetries, the dual theory has a remaining $\mathcal{N}=1$ supersymmetry left.

We demonstrated that the pullback of $C_{(8)}$ for values of $\phi$ that are odd fractions of $\pi$ is non-zero. This is important since previous papers such as \cite{Albash:2011nw} and \cite{Evans:2005ti}\footnote{Their $\phi$ is shifted compared to ours, hence their sine is our cosine, and vice versa.} argued that the pullback of $C_{(8)}$ is vanishing while using the Einstein metric frame. In particular, the embedding in \cite{Albash:2011nw} differs from ours only in the WZ term, and their result has an extra logarithmic divergence. The D7-brane embedding of \cite{Evans:2005ti} is quite different from ours, as it wraps non-trivially the deformed sphere. The missing WZ term in principle contributes to their IR potential, hence potentially affecting their conclusion too. Our supersymmetric condition method does not require the D-brane Lagrangian, so that the fulfilment of the equation of motion at the solution provides a strong proof that the WZ term is there, besides our explicit computations of the fluxes. We can conclude also that the string frame is the right metric frame to use in the D-brane analysis, instead of the Einstein frame.


A rather surprising result we found is that the renormalized action for our embedding is a non-trivial function of the quark mass.
As a consistency check, our D7-brane configuration reduces to the known solution in the AdS/CFT case \cite{Karch:2005ms}, close to the boundary of the Pilch-Warner geometry. At this limit, the renormalized action indeed vanishes, and hence we can conclude that the chiral condensate is zero. It is definitely interesting to understand the general case, and see whether the dual $\mathcal{N}=1$ theory has indeed a non-vanishing condensate or not. It is suggested in the literature that chiral symmetry in zero temperature $\mathcal{N}=1$ SYM can be spontaneously broken. We will leave this investigation for future works.


As another possible future work, the fluctuations around our D-brane, corresponding to the meson spectrum on the field theory side, could be studied.
It would also be interesting to consider the scenario of a non-vanishing gauge field on the deformed sphere. For example, in $AdS_5 \times S^5$, \cite{Kruczenski:2003be} studied mesons that carry angular momentum. This scenario has also been studied by \cite{Karch:2015vra} for the global $AdS_5 \times S^5$, where different topological inequivalent solutions were found. One could also review the thermal case studied in \cite{Albash:2011dq}, for which the geometry was derived in \cite{Buchel:2003ah}.





