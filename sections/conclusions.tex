\section{Conclusion}

In this paper, we found an exact supersymmetric D7-brane embedding in the Pilch-Warner geometry by solving the supersymmetric condition. In the field theory side, this setting corresponds to having a quark sector in the $\mathcal{N}=2^*$ SYM at zero temperature. Our D7-brane configuration reduces to the known solution in the AdS/CFT case \cite{Karch:2005ms}, in the near boundary limit. 

Although fundamental flavordynamics in $\mathcal{N}=2^*$ SYM in the holographic picture has been studied before, by \cite{Albash:2011nw} and briefly by \cite{Evans:2005ti}, they mistakenly deemed the pullback of $C_{(8)}$ to be zero. Consequently, the asymptotic solution that \cite{Albash:2011nw} found, which has a logarithmic term that the authors themselves found surprising, is simply not there. Our solution is exact and simplify their analysis, for example, the quark condensate is just zero, a result that is compatible with supersymmetry. Both authors from \cite{Albash:2011nw} and \cite{Evans:2005ti} argued within the Einstein frame of metric. It is indeed a subtlety that is not present for the simpler $AdS_5 \times S^5$, since the Einstein frame coincides with the string frame, due to a trivial dilaton. 

Our supersymmetric condition method does not require the WZ term, hence the fulfilment of the equation of motion at the solution provides a strong proof of that the WZ term is non-zero, besides our explicit proof. This also means that the string frame is the right metric frame to use. 

As a possible future work, we can study the fluctuations around our D-brane, hence the meson spectrum in the field theory side. 
It is also interesting to consider non-vanishing gauge fields on the deformed sphere, as done in \cite{Kruczenski:2003be} for the $AdS_5 \times S^5$ case, where their mesons carry angular momentum. One can also review the thermal case studied in \cite{Albash:2011dq}. Many probe branes is another direction, recently explored by \cite{Faedo:2019jlp} for $AdS_5 \times S^5$ with two branes.


