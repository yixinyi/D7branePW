\section{D7-brane}

We consider a probe D7-brane, whose worldvolume is induced from the target space with 
\begin{equation}
 \theta = \theta(c), \quad \phi=\frac{(2 n + 1)\pi}{2}.
\end{equation}
Therefore, the induced metric, from \eqref{eq:PWmetric} with $d\theta = \theta'(c) dc$ and $d\phi=0$, is:
\begin{align}\label{eq:PWmetric}
ds_{D7}^2 =
\Omega^2 dx_\mu dx^\mu 
- (V_c^2 +V_\theta^2 \theta'(c)^2)\, dc^2 - V_1^2 \sigma_1^2 - V_2^2 (\sigma_2^2 + \sigma_3^2),
\end{align}

Our goal is prove that such configuration exists by finding it explicitly using the supersymmetric condition. 



%%%%%%%%%%%%%%%%%%%%%%%%%%%%%%%%%%%%%%%%%%%%%%%%%%%%%%%%%%%%%%%%%%%%%%%%%%%%%%%%%%%%%%%%%%%%%%%%%
\subsection{Action}

With our particular choice of $\phi$, 
\begin{equation}
 P[B_{(2)}] = 0,
\end{equation}
and we assume no worldvolume gauge field, $F = 0$. 
\comment{In order to have a zero net charge? see KK. See 0304032 (Krucenski et al), where they have F !=0, their mesons carry angular momentum, so they consider gauge fields on S3, while we don't.}

Then the D7-brane action is simply
\begin{align}
 S & = -T_7 \int_\mathcal{M} d^8\xi \, e^{-P[\Phi] } \sqrt{-\det g} +
 T_7\int _\mathcal{M} P[C_{(8)}].
\end{align}

Note that papers \cite{Albash:2011nw} and \cite{Evans:2005ti} argued that the WZ sector is vanishing, however, they missed considering the dilaton term in the string frame for the metric, which effectively cancels the vanishing factor, leading to a finite value, as we will show. \comment{explain C8}.

More explicitly, the action is:
\begin{align}\label{eq:ActionWithTheta'}
 S = & -T_7 \int_\mathcal{M} d^8\xi \, \dfrac{A \cos^3\theta (c) \sqrt{c X_1(c, \theta(c))}}{\left(c^2-1\right)^3} \sqrt{\left(c^2-1\right)^2 A(c) \theta '(c)^2+c}\\
 &+
 T_7\int _\mathcal{M} d^8\xi \, \dfrac{A^2 \cos^4\theta(c)}{4 \left(c^2-1\right)^2}.
\end{align}

% \dfrac{A^2 \sin(\theta(c)) \cos^3(\theta(c) )}{\left(c^2-1\right)^2} % d[C8]


%%%%%%%%%%%%%%%%%%%%%%%%%%%%%%%%%%%%%%%%%%%%%%%%%%%%%%%%%%%%%%%%%%%%%%%%%%%%%%%%%%%%%%%%%%%%%%%%%
\subsection{Kappa symmetry projector}

The kappa symmetry projector for our configuration is:
\begin{align}
\Gamma = - \dfrac{ \gamma_{(8)} \mathcal{I} }{\sqrt{-\det g}},
\end{align}
with
\begin{align}
 \gamma_{(8)} = - V_x^4 V_1 V_2^2 \Gamma_{1 2 3 4 7 8 9}( V_c \Gamma_5 +  V_{\theta} \theta'(c) \Gamma_6), 
\end{align}
where we used capital gammas to denote de gamma matrices in the local frame. 


%%%%%%%%%%%%%%%%%%%%%%%%%%%%%%%%%%%%%%%%%%%%%%%%%%%%%%%%%%%%%%%%%%%%%%%%%%%%%%%%%%%%%%%%%%%%%%%%%
\subsection{Supersymmetric condition}

For the Killing spinor \eqref{eq:KillingSpinor}, the invertible operator in \eqref{eq:susyCondition0} is:
\begin{align}
 \mathcal{O} &= \exp{\left(\frac{\alpha}{2}\Gamma_{56} \right)} \exp{\left(-\frac{\phi}{2}\, \Gamma_{610} \right)} \exp{\left(\frac{\beta}{2}\Gamma_{710} \mathcal{K} \right)} \\
 \mathcal{O}^{-1} &=  \exp{\left(-\frac{\beta}{2}\Gamma_{710} \mathcal{K} \right)} 
 \exp{\left(\frac{\phi}{2}\, \Gamma_{610} \right)} 
 \exp{\left(-\frac{\alpha}{2}\Gamma_{56} \right)}.
\end{align}


The kappa symmetry projector contains the operator $\mathcal{I}$, which can be replaced as follows (see notation in \ref{sec:KillingSpinor}):
\begin{equation}
 \mathcal{I}\eta =-i \eta =\eta  \Gamma_{610},
\end{equation}
where we used $\mathcal{P}_+ \eta =\eta$. 

Then, \eqref{eq:susyCondition0} reduces to:
\begin{equation}
 \Pi_{-} \mathcal{O}^{-1} \gamma_{(8)} \mathcal{O} \Gamma_{610} \Pi_{+}  = 0.
\end{equation}
which, after manipulating the gamma matrices, gives:
\begin{equation}
-i V_1 V_2^2 V_x^4 \Pi_+ \Gamma_{6 8 9 10} \mathcal{K} \sin\beta \left(V_c \sin\alpha - V_\theta \cos\alpha \, \theta'(c)\right) = 0
\end{equation}
Therefore, the condition our configuration must satisfy in order to preserve supersymmetry is:
\begin{equation}\label{eq:susyConditionTheta}
 \theta'(c) = \dfrac{V_c}{V_\theta} \tan\alpha = \dfrac{c \, \tan\theta(c)}{c^2-1} .
\end{equation}


One can repeat the analysis of \eqref{eq:susyCondition0} for the other projector $\mathcal{P}_{\pm}$, and it will give the same condition \eqref{eq:susyConditionTheta}.

\subsection{Equation of motion}

One can also easily check that the equation of motion from the action \eqref{eq:ActionWithTheta'} is fulfilled with \eqref{eq:susyConditionTheta}.



\subsection{Solution}
The solution to the differential equation \eqref{eq:susyConditionTheta} is:
\begin{equation}\label{eq:susyConditionSolution}
\sin\theta(c) = \text{const} \sqrt{c^2-1}.
\end{equation}


%%%%%%%%%%%%%%%%%%%%%%%%%%%%%%%%%%%%%%%%%%%%%%%%%%%%%%%%%%%%%%%%%%%%%%%%%%%%%%%%%%%%%%%%%%%%%%%%%
\subsection{Discussion}

Mass of the quarks. 

projector at solution

Action, holo-rg

Geometry of D7, show plots

Limit to known AdSxS case solution, done in \cite{Karch:2005ms}.
