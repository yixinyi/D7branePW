\section{D7-brane solution}

We consider a probe D7-brane, whose worldvolume is induced from the target space with 
\begin{equation}
 \theta = \theta(c), \quad \phi=\frac{(2 n + 1)\pi}{2}.
\end{equation}
Therefore, the induced metric, from \eqref{eq:PWmetric} with $d\theta = \theta'(c) dc$ and $d\phi=0$, is:
\begin{align}\label{eq:PWmetric}
ds_{D7}^2 =
\Omega^2 dx_\mu dx^\mu 
- (V_c^2 +V_\theta^2 \theta'(c)^2)\, dc^2 - V_1^2 \sigma_1^2 - V_2^2 (\sigma_2^2 + \sigma_3^2),
\end{align}

Our goal is prove that such configuration exists by finding it explicitly using the supersymmetric condition. 
We will also check that it solves the equation of motion derived from the D-brane action.

\subsection{Action}


With our particular choice of $\phi$, 
\begin{equation}
 P[B_{(2)}] = 0.
\end{equation}

We will argue now that a vanishing worldvolume gauge field $F=0$ is compatible with our ansatz, through its equation of motion.

The WZ action, then, contains the following terms:
\begin{equation}
 \int F^2 \wedge P[C_{(4)}], \quad \int F\wedge P[C_{(6)}], \quad \int P[C_{(8)}].
\end{equation}
The first term above vanishes because $F^2=0$, due to antisymmetry. The second term above vanishes because $P[C_{(6)}]=0$, see its expression in the appendix. The only remaining term that is non-zero is 8-form term. 
This is in contrary to what Albash and Johnson, whose argument fails considering the dilaton term in the string frame that will cancel the cosine term.

Since the WZ term contains no gauge field $F$, its equation of motion:
\begin{equation}
 \dfrac{\pd L_{DBI}}{\pd F_{ij}} = 0,
\end{equation}
is solved by setting $F=0$. 

Then,
\begin{align}
 S & = -T_7 \int_\mathcal{M} d^8\xi \, e^{-P[\Phi] } \sqrt{-\det g} \pm
 T_7\int _\mathcal{M} P[C_{(8)}],
\end{align}

\subsection{Kappa symmetry projector}

The kappa symmetry projector for our configuration is:
\begin{align}
\Gamma = - \dfrac{ \gamma_{(8)} \mathcal{I} }{\sqrt{-\det g}},
\end{align}
with
\begin{align}
 \gamma_{(8)} = - V_x^4 V_1 V_2^2 \Gamma_{1 2 3 4 7 8 9}( V_c \Gamma_5 +  V_{\theta} \theta'(c) \Gamma_6), 
\end{align}
where we used capital gammas to denote de gamma matrices in the local frame. 


\subsection{Supersymmetric condition}
For the Killing spinor \eqref{eq:KillingSpinor}, we will compute the following version of the supersymmetric condition \eqref{eq:susyCondition0}:
\begin{equation}
 \Pi_{-} \mathcal{O}_{-\beta} \mathcal{O}_{-\phi}\mathcal{O}_{-\alpha} \Gamma \mathcal{O}_{\alpha} \mathcal{O}_{\phi} \mathcal{O}_{\beta} \Pi_{+}  = 0.
\end{equation}

The kappa symmetry projector contains the operator $\mathcal{I}$, which can be replaced as follows (see notation in \ref{sec:KillingSpinor}):
\begin{equation}
 \mathcal{I}\eta =-i \eta =\eta  \Gamma_{6,10},
\end{equation}
where we used $P_+$. 

Then,
\begin{equation}
 \Pi_{-} \mathcal{O}_{-\beta} \mathcal{O}_{-\phi}\mathcal{O}_{-\alpha} \gamma_{(8)} \mathcal{O}_{\alpha} \mathcal{O}_{\phi} \mathcal{O}_{\beta} \Gamma_{6,10} \Pi_{+}  = 0.
\end{equation}
which, after some manipulations with the gamma matrices, reduces to:
\begin{equation}
-i V_1 V_2^2 V_x^4 \Pi_+ \Gamma_{6 8 9 10} \mathcal{K} \sin\beta \left(V_c \sin\alpha - V_\theta \cos\alpha \, \theta'(c)\right) = 0
\end{equation}
Therefore, the condition our configuration must satisfy in order to preserve supersymmetry is:
\begin{equation}
 \boxed{\theta'(c) = \dfrac{V_c}{V_\theta} \tan\alpha}.
\end{equation}

One can repeat the analysis of \eqref{eq:susyCondition0} for the other projector $\mathcal{P}_{\pm}$, and it will give the same condition. 





\subsection{Equation of motion}

\begin{equation}
 L_{DBI} = \frac{\sqrt{c} A(c) \cos ^3(\theta (c)) \sqrt{\text{X1}(c,\theta (c))} \sqrt{\left(c^2-1\right)^2 A(c) \theta '(c)^2+c}}{\left(c^2-1\right)^3}
\end{equation}

The equation of motion for $\theta(c)$,
\begin{equation}
\dfrac{\pd L}{\pd \theta(c)} - \dfrac{\pd}{\pd c} \dfrac{\pd L}{\pd \theta'(c)} = 0,
\end{equation}
gives



\begin{align*}
% 
e^{-P[\Phi]} &=\sqrt{\frac{X_2}{c X_1}}
\end{align*}
