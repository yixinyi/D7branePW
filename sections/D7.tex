\section{D7-brane solution}

We consider a probe D7-brane, whose worldvolume is induced from the target space with 
\begin{equation}
 \theta = \theta(c), \quad \phi=\frac{(2 n + 1)\pi}{2}.
\end{equation}
Therefore, the induced metric, from \eqref{eq:PWmetric} with $d\theta = \theta'(c) dc$ and $d\phi=0$, is:
\begin{align}\label{eq:PWmetric}
ds_{D7}^2 =
\Omega^2 dx_\mu dx^\mu 
- (V_c^2 +V_\theta^2 \theta'(c)^2)\, dc^2 - V_1^2 \sigma_1^2 - V_2^2 (\sigma_2^2 + \sigma_3^2),
\end{align}

Our goal is prove that such configuration exists by finding it explicitly using the supersymmetric condition. 
We will also check that it solves the equation of motion derived from the D-brane action.

\subsection{Action}


The particular choice of $\phi$ will make many terms vanishing, as shown in AlbashJohnson, however, the $C_8$ term is non-zero, in contrary to what they say. 
Albash and Johnson's argument fails considering the dilaton term in the string frame that will cancel the cosine term.

\begin{align*}
P[B] & = 0 \\
% 
e^{-P[\Phi]} &=\sqrt{\frac{X_2}{c X_1}}\\
%
P[C_{(4)}] &= 4 \, \omega \, dx_0\wedge dx_1\wedge dx_2\wedge dx_3 \\
%
P[C_{(6)}] &= 0\\
%
P[C_{(8)}] &=
 \left\{
-\frac{\left[\left(\left(c^2+3\right) A-4 c\right) \cos^2\theta -2 A\right]A \, a_3^2 }{4 \left(c^2-1\right)^3}+\frac{((c^2-1) A+4 c)A \cos^4\theta}{8 c^2 \left(c^2-1\right)^2}
\right.\\
&\left.
-\frac{\left(A^2 \sin^2\theta +2 c A+c^2 \cos^2\theta \right) A^2 \sin^3\theta \cos^3\theta \, \theta'}{2 c \left(c^2-1\right)  X_2^2}
\right\}\\
&\sigma_1\wedge \sigma_2\wedge \sigma_3\wedge dc\wedge dx_0\wedge dx_1\wedge dx_2\wedge dx_3
\end{align*}

Since the WZ term contains no gauge field $F$, its equation of motion:
\begin{equation}
 \dfrac{\pd L_{DBI}}{\pd F_{ij}} = 0,
\end{equation}
is solved by setting $F=0$. 

Then,
\begin{align}
 S & = -T_7 \int_\mathcal{M} d^8\xi \, e^{-P[\Phi] } \sqrt{-\det g} \pm
 T_7\int _\mathcal{M} P[C_{(8)}],
\end{align}

The equation of motion for $\theta(c)$ is:
\begin{equation}
\dfrac{\pd L}{\pd \theta(c)} - \dfrac{\pd}{\pd c} \dfrac{\pd L}{\pd \theta'(c)} = 0
\end{equation}

For our values of $\phi$ and $F=0$:
\begin{equation}
 L_{DBI} = \frac{\sqrt{c} A(c) \cos ^3(\theta (c)) \sqrt{\text{X1}(c,\theta (c))} \sqrt{\left(c^2-1\right)^2 A(c) \theta '(c)^2+c}}{\left(c^2-1\right)^3}
\end{equation}





\subsection{Kappa symmetry projector}

\subsection{Supersymmetric condition}

