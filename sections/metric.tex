The ten dimensional Pilch-Warner metric in Einstein frame is parametrized in the following way, see also \cite{Buchel:2000cn}:
\begin{align}\label{eq:PWmetric}
ds^2 =
\Omega^2 dx_\mu dx^\mu -\left(
V_c^2 dc^2 + V_\theta^2 d\theta^2 + V_1^2 \sigma_1^2 + V_2^2 (\sigma_2^2 + \sigma_3^2) + V_\phi^2 d\phi^2\right),
\end{align}
with $c>1$ and the range for the angles being: $ 0\leq \theta \leq\pi/2, 0\leq \phi \leq 2\pi$. 

The various coefficients are functions of $c$ and the angle $\theta$, and are given by:
\begin{align}\label{eq:PWvielbeins}
\Omega &= \frac{c^{1/8} A^{1/4} X_1^{1/8} X_2^{1/8}}{(c^2 - 1)^{1/2}},\nonumber\\
V_c &= \frac{c^{1/8}X_1^{1/8} X_2^{1/8}}{A^{3/4} (c^2-1)},\nonumber\\
V_\theta &= \frac{X_1^{1/8} X_2^{1/8}}{c^{3/8}A^{1/4}},\nonumber\\
V_1 &= \frac{A^{1/4}X_1^{1/8} }{c^{3/8}X_2^{3/8}} \cos\theta,\nonumber\\
V_2 &= \frac{c^{1/8}A^{1/4}X_2^{1/8} }{X_1^{3/8}} \cos\theta, \nonumber\\
V_\phi &= \frac{c^{1/8}X_1^{1/8} }{A^{1/4}X_2^{3/8}} \sin\theta,
\end{align}
and
\begin{align}
X_1 &=  \cos^2\theta + cA  \sin^2\theta,\nonumber\\
X_2 &= c \cos^2\theta + A  \sin^2\theta, \nonumber\\
A &= c+(c^2 -1)\frac{1}{2}\ln\left(\frac{c-1}{c+1}\right).
\end{align}

The deformed 3-sphere is parametrized by the $SU(2)$ left invariant forms, i.e. the Maurer-Cartan forms:
\begin{equation}
\sigma_i = \tr(g^{-1}\tau_i d g), \quad i = 1,2,3
\end{equation}
where $\tau_i$ are the Pauli matrices and $g$ is a group element of SU(2). The 1-forms satisfy the relation\footnote{Other conventions might introduce an extra global sign, for example in \cite{Buchel:2000cn}.}
\begin{equation}
 d\sigma_i  = \epsilon_{i j k} \sigma_j \wedge \sigma_k.
\end{equation}
We do not need to explicitly parametrize these forms for the purpose of the present paper, however, for the interested readers, there is an example using Euler angles in the appendix of \cite{Chen-Lin:2015xlh}. 


\subsection{String frame}

For all the purposes of the present paper, we will use the metric in the string frame, which is achieved by a general conformal transformation, or in other words:
\begin{equation}
 ds^2_\text{Einstein} = e^{-\Phi / 2} ds^2_\text{string},
\end{equation}
where $\Phi$ is the dilaton.


\subsection{The near boundary geometry}

The boundary of the Pilch-Warner geometry is located at $c = 1$. 
Close to the boundary, $c \approx 1 + \frac{z^2}{2}$, with $z$ small, we recover the $AdS_5 \times S^5$ geometry in Poincare coordinates and the Hopf parametrization for $S^5$:
\begin{equation*}
ds^2=\dfrac{dx^{\mu } dx_{\mu}-dz^2}{z^2}-\left(d\theta^2+\cos^2\theta \left(\sigma_1^2+\sigma_2^2+\sigma_3^2\right)+\sin^2\theta \,d\phi^2 \right).
\end{equation*}
