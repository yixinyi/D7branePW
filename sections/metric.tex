The Pilch-Warner metric, in Einstein frame\footnote{The Einstein frame and the string frame are related by a general conformal transformation: $ds^2_{Einstein} = e^{-\Phi / 2} ds^2_{string}$, where $\Phi$ is the dilaton.}, can be written as \cite{Pilch:2003jg}
\begin{align}\label{eq:PWmetric}
ds^2 =
\Omega^2 dx_\mu dx^\mu -\left(
V_r^2 dr^2 + V_\theta^2 d\theta^2 + V_1^2 \sigma_1^2 + V_2^2 (\sigma_2^2 + \sigma_3^2) + V_\phi^2 d\phi^2\right),
\end{align}
where the various coefficients are functions of the radial coordinate $r$ and the angle $\theta$, and are given by
\begin{align}\label{eq:PWvielbeins}
\Omega &= \frac{c^{1/8} A^{1/4} X_1^{1/8} X_2^{1/8}}{(c^2 - 1)^{1/2}},\nonumber\\
V_r &= \frac{c^{1/8}X_1^{1/8} X_2^{1/8}}{A^{1/12}},\nonumber\\
V_\theta &= \frac{X_1^{1/8} X_2^{1/8}}{c^{3/8}A^{1/4}},\nonumber\\
V_1 &= \frac{A^{1/4}X_1^{1/8} }{c^{3/8}X_2^{3/8}} \cos\theta,\nonumber\\
V_2 &= \frac{c^{1/8}A^{1/4}X_2^{1/8} }{X_1^{3/8}} \cos\theta, \nonumber\\
V_\phi &= \frac{c^{1/8}X_1^{1/8} }{A^{1/4}X_2^{3/8}} \sin\theta,
\end{align}
and
\begin{align}
X_1 = & \cos^2\theta + cA  \sin^2\theta,\nonumber\\
X_2 = & c \cos^2\theta + A  \sin^2\theta.
\end{align}
$A$ and $c$ are functions of the radial coordinate $r$, which satisfy
\begin{align}
A = c+(c^2 -1)\frac{1}{2}\ln\left(\frac{c-1}{c+1}\right),
\end{align}
and
\begin{align}
\label{dcdr} \frac{dc}{dr} &= - A^{2/3}(c^2 - 1),\\ 
\label{dAdr} \frac{dA}{dr} &= 2 A^{2/3}\left(1 - c A\right).
\end{align}

The deformed 3-sphere is parametrized by the $SU(2)$ left invariant forms, i.e. the Maurer-Cartan forms:
\begin{equation}
\sigma_i = \tr(g^{-1}\tau_i d g), \quad i = 1,2,3
\end{equation}
where $\tau_i$ are the Pauli matrices and $g$ is a group element of SU(2). The 1-forms satisfy the relation\footnote{Other conventions might introduce an extra global sign, for example in \cite{Buchel:2000cn}.}
\begin{equation}
 d\sigma_i  = \epsilon_{i j k} \sigma_j \wedge \sigma_k.
\end{equation}
We need no explicit parametrization of these forms in this paper, however, for interested readers, there is an example using Euler angles in the appendix of \cite{Chen-Lin:2015xlh}.


\subsection{Using $c$ as a coordinate}

In this paper, as in \cite{Buchel:2000cn}, we will use $c$ as coordinate instead of $r$, via the transformation \eqref{dcdr}. 
In this way, we need not solve \eqref{dAdr}. 
Therefore, the radial part of the metric \eqref{eq:PWmetric}, i.e. $V_r^2 dr^2$, is replaced by $V_c^2 dc^2$, where
\begin{equation}
 V_c = \frac{c^{1/8}X_1^{1/8} X_2^{1/8}}{A^{3/4} (c^2-1)}.
\end{equation}

Notice the negative sign in \eqref{dcdr}, which means that this change of coordinate involves a parity transformation. This will have consequences in e.g. Hodge dual, the gamma matrices.





