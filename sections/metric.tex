We parametrize the ten dimensional Pilch-Warner metric in the following way:
\begin{align}\label{eq:PWmetric}
ds^2 =
\Omega^2 dx_\mu dx^\mu -\left(
V_c^2 dc^2 + V_\theta^2 d\theta^2 + V_1^2 \sigma_1^2 + V_2^2 (\sigma_2^2 + \sigma_3^2) + V_\phi^2 d\phi^2\right),
\end{align}
with $\mu=1,\ldots,4$ and the coordinates: $c\in(1, \infty), \theta \in [0, \pi/2], \phi \in [0, 2\pi]$. 

The various coefficients are functions of $c$, $\theta$ and $\phi$, where the latter dependence comes only from the dilaton $\Phi$ prefactor\footnote{The dilaton factor comes from the fact that we are using the metric in the string frame. In the Pilch-Warner literature, often the metric in the Einstein frame is shown. Both frames are related by a general conformal transformation, i.e. $ds^2_\text{Einstein} = e^{-\Phi / 2} ds^2_\text{string}$.}, with its explicit form shown in the next subsection. 
The coefficients are given by:
\begin{align}\label{eq:PWvielbeins}
\Omega &=e^{-\Phi/4} \frac{c^{1/8} A^{1/4} X_1^{1/8} X_2^{1/8}}{(c^2 - 1)^{1/2}},\nonumber\\
V_c &= e^{-\Phi/4}\frac{c^{1/8}X_1^{1/8} X_2^{1/8}}{A^{3/4} (c^2-1)},\nonumber\\
V_\theta &= e^{-\Phi/4}\frac{X_1^{1/8} X_2^{1/8}}{c^{3/8}A^{1/4}},\nonumber\\
V_1 &= e^{-\Phi/4}\frac{A^{1/4}X_1^{1/8} }{c^{3/8}X_2^{3/8}} \cos\theta,\nonumber\\
V_2 &= e^{-\Phi/4}\frac{c^{1/8}A^{1/4}X_2^{1/8} }{X_1^{3/8}} \cos\theta, \nonumber\\
V_\phi &= e^{-\Phi/4}\frac{c^{1/8}X_1^{1/8} }{A^{1/4}X_2^{3/8}} \sin\theta,
\end{align}
and
\begin{align}
X_1 &=  \cos^2\theta + cA  \sin^2\theta,\nonumber\\
X_2 &= c \cos^2\theta + A  \sin^2\theta, \nonumber\\
A &= c+(c^2 -1)\frac{1}{2}\ln\left(\frac{c-1}{c+1}\right).
\end{align}

The deformed 3-sphere is parametrized by the $SU(2)$ left invariant forms, i.e. the Maurer-Cartan forms:
\begin{equation}
\sigma_i = \tr(g^{-1}\tau_i d g), \quad i = 1,2,3
\end{equation}
where $\tau_i$ are the Pauli matrices and $g$ is a group element of SU(2). The 1-forms satisfy the relation\footnote{Other conventions might introduce an extra global sign, for example in \cite{Buchel:2000cn}.}
\begin{equation}
 d\sigma_i  = \epsilon_{i j k} \sigma_j \wedge \sigma_k.
\end{equation}
We do not need to explicitly parametrize these forms for the purpose of the present paper, however, for the interested readers, there is an example using Euler angles in the appendix of \cite{Chen-Lin:2015xlh}. 


\subsection{Local frame}\label{sec:localframe}
The non-coordinate basis, also known as the local frame, is specified by the Minkowski metric $\eta_{a b}$. It is related to the curved space metric $g_{M N}$ via vielbeins, according to:
\begin{equation*}
 g_{M N} = e^a_M e^b_N \eta_{a b}.
\end{equation*}
In our case, the metric \eqref{eq:PWmetric} is diagonal\footnote{Once we use the explicit parametrization for the deformed sphere, it is not diagonal.}, hence the vielbeins and the inverse vielbeins are precisely the coefficients \eqref{eq:PWvielbeins} and its inverse, respectively.

When we handle the curved-space gamma matrices $\gamma_{M}$, we will go to the local frame in order to use the constant $\Gamma_a$ matrices:
\begin{equation*}
 \gamma_{M} = e^a_M \Gamma_{a}.
\end{equation*}




\subsection{The near boundary geometry}

The boundary of the Pilch-Warner geometry is located at $c = 1$. 
Close to the boundary, $c \approx 1 + \frac{z^2}{2}$, with $z$ small, we recover the $AdS_5 \times S^5$ geometry in Poincare coordinates and the Hopf parametrization for $S^5$:
\begin{equation}\label{eq:AdS5xS5metric}
ds^2=\dfrac{dx^{\mu } dx_{\mu}-dz^2}{z^2}-\left(d\theta^2+\cos^2\theta \left(\sigma_1^2+\sigma_2^2+\sigma_3^2\right)+\sin^2\theta \,d\phi^2 \right).
\end{equation}
